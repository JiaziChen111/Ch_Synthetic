\documentclass[12pt]{article}
\usepackage[utf8]{inputenc}					  % Input accented characters
\usepackage{underscore}						   % Control the behaviour of "_" in the text
%\usepackage[T1]{fontenc}					  % Fonts to use for printing characters
\usepackage[margin=1in]{geometry} 	   % Sets the margins of the file
\usepackage[colorlinks=true]{hyperref} 	% colorlinks=true gets rid of the awful boxes
\usepackage{enumitem,amssymb}
\usepackage{xcolor}
	\definecolor{clink}{rgb}{0,0,1} 			 % Blue
	\definecolor{ccite}{rgb}{0,0.3,0.9} 	   % Light blue
	\definecolor{curl}{rgb}{0.3,0,0.9} 			% Red blue
	\hypersetup{linkcolor={clink}, citecolor={ccite}, urlcolor={curl}} 		% Internal links, Citations, External links/urls

% Link to ToC from section
\newcommand{\gototoc}{\vspace{-1.8cm} \null\hfill [\hyperlink{toc}{Go to ToC}] \newline}

\newlist{todolist}{itemize}{2}					 % To-do list with checkmarks
\setlist[todolist]{label=$\square$}
\usepackage{pifont}
	\newcommand{\cmark}{\ding{51}}
	\newcommand{\xmark}{\ding{55}}
	\newcommand{\done}{\rlap{$\square$}{\raisebox{2pt}{\large\hspace{1pt}\cmark}}%
	\hspace{-2.5pt}}
	\newcommand{\wontdo}{\rlap{$\square$}{\large\hspace{1pt}\xmark}}

\begin{document}
	\title{Tasks for `Emerging~Markets Sovereign Yields and U.S. Monetary Policy
%Decomposing the Sovereign Yield Curves of Emerging~Markets
%Decomposing the Yield Curves of Emerging~Markets
%Bond Risk Premia in Emerging Markets: Dynamics, Comovement and Drivers
%Comovement of the Sovereign Yields of Emerging~Markets
%Do the Sovereign Yields of Emerging Markets ?
%Comovement of the Sovereign Yields of Emerging Markets: The Role of Synthetic Yield Curves
%Comovement of Local Currency Sovereign Yields: The Role of Synthetic Yield Curves
%Term Premia in Emerging Markets
'}
	\author{M. Pavel Solís M.}
	\date{}
	\maketitle
	\hypertarget{toc}{}								 % Allows to link back to the ToC
	\tableofcontents
	\vspace{2.5\bigskipamount}
	

\section{Data: BLP}
\gototoc
\begin{todolist}
	\item After comparing datasets downloaded from BLP on Mar 2018 and Feb 2019: BLP updated values (slightly) in variables starting on: 9/30/2009 (USSW), 5/4/11 (USBC, USBA).
	\item EUBS1-EUBS20 for PLN deleted since already downloaded for HUF.
	\item Check PZSW*V3 Curncy against PZSW* Curncy for PLN because the former have a shorter history (became active in 2013 vs 2000). Why the substitution? Was it because PZSW* Curncy were discontinued?
	\item Download PZSW15 and Curncy PZSW20 to complement the history 6/12/2000-1/30/2013 for Curncy PZSW15V3 Curncy and PZSW20V3 Curncy, respectively.
	\item Check MRBS* starting in 4/20/2016 because in the data downloaded in 2019 they were flat most of the time after that date.
	\item Check the following series b/c they remain mostly flat during those periods: SABS15 Curncy b/w 4/15/2014-1/27/2016, SABS20 Curncy b/w 4/15/2014-8/28/2015, SABS25 Curncy b/w 1/3/2000-9/5/2011, SABS30 Curncy b/w 1/3/2000-5/10/2010.
	\item Check constant yield curves.
	\begin{todolist}
		\item NOK: C26610Y Index is flat after Nov 2012, C26610Y is flat twice during the sample.
		\item PHP: Since 29-Oct-2018 BFV data does not vary. In the dataset, the BFV series includes IYC (converted into CE par) yields since 1-Oct-2018. Update BFV series w/ BLP data when access to it is available.
	\end{todolist}
	\item Include variables in datasets: CDS, longest histories in consumption and inflation for EMS. Same macro-financial data for AEs than for EMs?
\end{todolist}

\section{Datasets: Daily}
\gototoc
\begin{todolist}
	\item For BRL RHO 2Y deviates from DIS starting in 2016. RHO 7Y is not available relative to DIS.
	\item For GBP RHO 7Y increases a lot relative to DIS in 2010.
	\item For PEN RHO 7Y big decrease in 2018.
	\item For PHP RHO 7Y big spike in 2010.
	\item For THB, AUD, CHF, DKK, EUR, NZD, SEK RHO 7Y and CAD 3Y, 7Y big declines.
	\item Review RHO: 
	\begin{todolist}
		\item COP 1Y flat b/w 2012-2015.
		\item HUF 3M.
		\item IDR 1Y.
		\item ILS 3M, 7Y.
		\item MYR 2Y.
	\end{todolist}
	\item[\done] For NOK RHO 2Y, 3Y, 7Y have a hole in 2008-2009. Reason: cutoff date of convention to compute IRS.
\end{todolist}

\section{Data: Codes}
\begin{todolist}
	\item Consider shorter histories of some series within a country.
	\begin{todolist}
		\item For PLN consider removing PZBSEC6 Curncy, PZBSEC8 Curncy, PZBSEC9 Curncy (and PZSW6V3 Curncy, PZSW8V3 Curncy, PZSW9V3 Curncy) because they start in 2011 and that seems to eliminate the data before 2011 for PLN even if it is available for all the other tenors; plus it may not be needed, it depends on the other variables in the formula having the 6Y tenor available. Or modify the code so that those tenors are considered once they become available (and discarded before 2011).
		\item Availability of data for PZBSEC6 Curncy and PZSW15 and Curncy might be related to the history of the synthetic curves for PLN.
		\item Check that the shorter history of PZSW15V3 Curncy and PZSW20V3 Curncy (starts in 2013) does not affect the history of the synthetic curves for PLN.
		\item Check that MRBS10 and lower start around 2005, while MRBS15 around 2007 and MRBS20 in mid-2009.
		\item RRUSSW10 and lower start in 2006, while RRUSSW15 and RRUSSW20 in 2009.
		\item TBSWNI10  and lower start in 2007, TBSWNI15 and TBSWNI20 in 2009.
	\end{todolist}
	\item[\done] In compute_fp_long.m address cases for HUF, PLN and NOK. If BS (HUF, PLN) and IRS (NOK) series are NaN, use zeros.
	\item[\done] There are countries with BFV and IYC curves, transform BFV into ZC yields and compare the curves. BFV has CE par yields, IYC has CE ZC yields. Need to be converted into ZC continuously compounding.
	\item[\wontdo] For countries for which DIS have a longer history than me, check whether it is b/c we use different LC YC (BFV vs. IYC) and one has a longer history. For HUF and PLN is b/c BS series start in 2007 in my sample.
	\item[\done] Update codes since EUBS1-EUBS20 for HUF and PLN were deleted because they were already downloaded for EUR (plus EUBS25 and EUBS30 are now included).
\end{todolist}


\section{Data: Files}
\gototoc
\begin{todolist}
	\item[\done] In the dataset, the BFV series for PHP includes IYC (converted into CE par) yields since 1-Oct-2018. 
	\item[\done] In CE_Forecasts.xlsx, input first CPI 10Y forecast for PHP equal to next value since original value is unrealistically low.
	\item[\done] In AE_EM_Curves_Tickers.xlsx change cutoff date for NOK. Official date 9/2/2008 only applies to 1Y, 5Y and 10Y tenors. New date 5/20/2009 applies to all tenors.
	\item[\done] Data: CRSP.
	\begin{todolist}
		\item[\done] Download “Riskfree Series (1-month and 3-month)”, “Riskfree Series (4, 13, and 26-week)”, Fama Term Structure data for 3- 6- 9- months. Compare them.
		\item[\done] Update read_usyc.m to read CRSP data for maturities $<$ 1Y.
		\item[\done] Compare results in LC spread and forward premium for 3M against DIS.
	\end{todolist}
	\item[\done] It seems that EUBS1-EUBS20 for HUF can be deleted from AE_EM_Curves_BDH.xlsx (codes referencing BS_EUR so far ---fltr4tickers and compute_fp_long--- updated). Need consistency because the Type column in sheet Identifiers is different for HUF (BS_EUR) than for EUR (BS). 
	\begin{todolist}
		\item[\done] Compare time series of EUBS1-EUBS20 for HUF and for EUR to make sure they are the same (i.e. downloaded correctly).
		\item[\done] Highlight in red columns DM-DV in original_Zero_Swap_Curves_Bloomberg.xlsx to indicate that those tickers were removed in current dataset.
		\item[\done] Comment after comparison: small differences between 2018 and 2019. No change was made to 2019 series. Same data when comparing EUBS1-EUBS20 for HUF and EUR.
		\item[\done] Delete BS_EUR in Type column in README sheet in AE_EM_Curves_Tickers.xlsx. Update Type columns in HUF and PLN sheets for EUBS1-EUBS20 from BS_EUR to BS.
		\item[\done] Update AE_EM_Curves_BDH.xlsx.
	\end{todolist}
	\item[\done] Change name of files to US_Yield_Curve_Data.xlsx, IMF_Country_Codes.xlsx, ISO_Currency_Codes.xlsx, BIS_CB_Policy_rates.xlsx, CIP_Data.xlsx and CIP_data.dta (original_ removed).
	\item[\done] Value for PHP CPI 10Y changed from 1.3 to 4.8 since it is unrealistically low.
	\item[\done] Update CE_Forecasts.xlsx: change order of groups (APCF, LACF, EECF), correct dates for EECF.
	\item[\done] Delete EUBS1-EUBS20 for PLN in both sheets (All, Tickers) of AE_EM_Curves_BDH.xlsx.
\end{todolist}

\section{Codes}
\gototoc
\begin{todolist}
	\item Check if the formula for the CCS of HUF needs to be updated in compute_fp_long.m as indicated in the AE_EM_Curves_Tickers.xlsx file (check the sign ---plus or minus--- for TBS6v3).
	\item[\done] Updates to Kfs: remove commented nargin for mu_x and mu_y.
	\item[\done] Move loadings4ylds to the temp folder.
	\item[\done] Update loadings4ylds: help, round maturities so that input as [0.25 1:5:10]
	\item[\done] Consider to include files as subfunctions.
	\begin{todolist}
		\item[\done] findNaN.m into remove_NaNcols.m.
		\item[\wontdo] matchtnrs.m into extract_vars.m. extract_vars.m is already long.
		\item[\wontdo] fltr4tickers.m into extract_vars.m. Filter also used by compare_ycs.m and zc_yields.m.
		\item[\wontdo] y_NS into y_NSS.m. They are used separately by zc_yields.m.
	\end{todolist}
	\item[\done] Add read_cip.m and iso2names.m in m-files called in read_data.m.
	\item[\done] Move old codes to Aux->Temp folder.
	\item[\wontdo] Check what the mat-files (ccs_data.mat, ccs_info.mat) contain to decide whether to move them to the Aux folder. Obtained with old codes.
	\item[\done] Filter SPT in THdt to see if the currency tickers are identified. Yes, actually curncs is now obtained directly from the headers table.
\end{todolist}

\section{Facts}
\gototoc
\begin{todolist}
	\item No EM was at the ZLB over the period.
	\item Countries w/ LCNOM YCs  of less than 10Y maturities:
	\begin{todolist}
		\item IDR b/w 25-Feb-2003 and 15-Apr-2003 max tenor was 8Y.
		\item KOR b/w 3-Jan-2000 and 30-Oct-2000 max tenor was 8Y.
		\item RUB b/w 04-Apr-2006 and 19-Jun-2007 max tenor was 7Y.
		\item TRY  b/w 17-Feb-2005 and 22-Feb-2005 max tenor was 3Y.
		\item TRY  b/w 22-Feb-2005 and 28-Jun-2010 max tenor was 5Y.
	\end{todolist}
	\item Advantage of LCSYNT (longer history) instead of LCNOM for: BRL, IDR, PHP, RUB, TRY. Disadvantage for COP, HUF, ILS, MXN, MYR, PEN, PLN, THB, ZAR.
	\item Since 2016, CHF and DKK negative YC up to 10Y and 7Y, respectively. Before GFC, close to ZLB.
\end{todolist}

\section{Tables}
\gototoc
\begin{todolist}
	\item Clean Tables folder.
	\item Upward sloping YCs? Averages.
	\item Compare TPsynt in EMs vs TPnom in AEs. Averages. Compare to current table 2. Compare w/ average YC slope. 
	\item Fit. RMSE. Averages per country. Nominal AEs and synthetic EMs (incl. ILS and ZAR).
	\item Compare TPs w/ empirical measures as in Curcuru et al. 2018. Compare against survey-based TP, and measures of risk and uncertainty.
	\item Compare ssb_yP vs CBP surveys. Similar to comparing against model-free TPsvy. Comment this in footnote.
	\item Decompose nominal yields. Averages per country. Averages before and after the GFC to see if relative importance of the components changed. Drivers of yields: Average and std of components (TPsynt, LCCS and yE).
	\item Feeling of data. Summary statistics of yield (Guimaraes) and survey data (inflation, GDP, short rate). 
	\item Term structure of TP. Averages per country.
	\item Correlation of BRP and TPnom. Averages of components (relative importance) per country. Before and after the GFC.
	\item Compare std of BRP vs std of AE TP. C-R (2002): risk premiums are far more volatile in emerging markets than in developed economies. Bech and Lengwiler (2011) find that central bank purchases of bonds not only reduced long-term interest rates, but also led to lower interest rate volatility. 
	\item t-test for LCCS = 0 and TPsynt = 0. LCCS = 0 is equivalent to test TPnom = TPsynt.
	\item Average and std of real interest rate to see if on average is the same across EMs and how much variability it has compared to nominal and synthetic yields. Correlation b/w real rates of EMs.
	\item Notes for tables.
\end{todolist}


\section{Research Questions}
\gototoc
\begin{todolist}
	\item Why TPs differ across EMs? Average of TPsynt per country, is there a large variation among them? b/w them and AEs? 
	\item Why TP and CR move as they do? Which asset pricing model is consistent with the data (TP, CR)? What explanations of bond risk premia are supported? Compensation for disagreement (increases quantity of risk), for economic pain (increases price of risk), for economic uncertainty about GDP growth and/or inflation (increases price of risk), for volatility risk (increases quantity of risk)? \item High inflation uncertainty leads to high TP?
	\item Has the importance of TP and CR changed after the GR? More role of CR after the GR since large fiscal debts?
	\item Did UMP spread abroad? He and McCauley (2013) and Turner (2014). QE's purpose was to decrease TP in AE, did the same happen to EMs? TP \(<\) 0?
	\item Why TP \(<\) 0? Change in importance of AD relative to AS shocks? Change in correlation b/w consumption and inflation and b/w bonds and stocks (so bonds become hedges of stocks)? Gourio and Ngo (2016), Pericoli (2018), Campbell, Pflueger and Viceira (2018), Campbell, Sunderam and Viceira (2016). Or financial integration story?
	\item Financial integration: Correlation b/w debt flows from TIC data and TP. FX reserves. The increase in inflows reflected in an increase in FX reserves, lower bond yields or both. Case of capital controls in Brazil.
	\item GFCy is different at the short than at the long end of the curve. Obstfeld (2015)
	\item Is there a risk channel for monetary policy spillovers? Are LT rates more highly correlated than ST rates? Policy rate disconnect for ST and/or LT? US spillovers work through changes in risk premia? Expected rate in EMs react to changes in USTP?
	\item International comparisons of expectations (yE). Subtract inflation expectations and see whether r* differs. Is there a single world real interest rate? Is the real interest rate the same for all EMs? Plot min-max intervals to see spread like Kalemli-Ozcan for policy rates.
	\item What is the relationship b/w the YC components and the forward premium?
	\item Decompose the forward exchange premium into: an expectations part and a risk premium. Clarida-Taylor (1997), Burnside et al (2007), Backus et al (1995). If LT TP are more closely linked, then tau_rho is closer to zero and thus UIP holds more closely for LT rates than for shorter ones.
	\item[\done] TPs across EMs: Are they idiosyncratic or is there a common factor driving them? PCA for all yields. PCA for ST rates. PCA for LT rates. No common component might imply segmented markets, capital controls, home bias.
\end{todolist}

\section{Analysis: Short Term}
\gototoc
\begin{todolist}
	\item Panel to compute the real interest rates.
	\begin{todolist}
		\item Run regressions.
		\item Compute embedded yE using iteration.
		\item Compare against current values.
		\item Data for r* and compute expectation for EM policy rate.
		\item Contrast real rates: panel vs with Duffee's suggestion.
	\end{todolist}
	\item Update do files to be ready when updated dataset is generated in Matlab.
	\item Update Matlab codes.
	\begin{todolist}
		\item Create folders (CRSP, etc.) within Data \(->\) Raw folder to improve structure and facilitate updates.
		\item Check whether functions in post-estimation analysis need to be updated (simplified) after the bsl field added to the structure.
		\item Check whether usyc_analysis.m (done before atsm_estimation.m), compare_pcs.m (when unbalanced panels were thought to be used) and compare_atsm_surveys.m (predates ts_* post-estimation functions) need to be moved to Temp folder.
		\item Update ts_analysis file to generate dataset.
		\item Allocate 6-10 surveys to 5-10 forward rate instead of the 10Y yield.
		\item Update all functions after daily2monthly (forecast_cbpol.m, append_surveys.m, atsm_estimation.m, assess_fit.m) after datasets renamed from c_, n_, s_ to mc_, mn_, ms_. After that, move daily2monthly to Temp folder and use daily2dymy.m.
		\item Potential solution to TP in AEs not being negative is likely due to using the PCs from jszLLK_KF (instead of PCs not smoothed).
		\item Read EPU indexes for AEs and update those for EMs.
	\end{todolist}
	\item Shift time for variables in Matlab.
	\begin{todolist}
		\item Why IDR has LCSYNT data before RHO data becomes available?
		\item Shift RHO only for maturities \(>= 1\).
		\item Calculate LCSYNT in Matlab and compare against sftsyn in Stata.
		\item Compare shift RHO, LCNOM in Matlab vs sftrho,sftnom.
	\end{todolist}
	\item Re-estimate parameters using best yE (panel or Duffee).
	\item Post-estimation analysis.
	\begin{todolist}
		\item Use global activity indexes for GFCy and C-P global factor.
	\end{todolist}
	\item Update dataset in Matlab.
	\begin{todolist}
		\item Update ts_analysis file to generate dataset.
		\item Include USYC KW decomposition (yP and TP) for 2Y and 10Y.
		\item Include 3 PCs and estimated pricing factors per country.
		\item Include EPU indexes for AEs to see how they correlate w/ AE TPnom.
		\item Include global PC and non-US common factor.
	\end{todolist}
	\item Reproduce LPs.
	\begin{todolist}
		\item Save figures to be used in pdf.
	\end{todolist}
	\item Additional stuff: TIC data, S-W inflation decomposition, EPU.
	\begin{todolist}
		\item Correlation of YC components with the 3 PCs and estimated pricing factors.
		\item LPs for effects of US MPS on the 3 pricing factors of EMs. Need to add daily level, slope and curvature for each country to dataset.
		\item Stock and Watson inflation decomposition. Follow Wright 2011.
		\item Skim through Warnock papers on TIC data.
		\item Include TIC data for each country (inflows and outflows) in dataset.
		\item Diebold-Yilmaz spillover index.
	\end{todolist}
	\item Read literature.
	\begin{todolist}
		\item Interpret the results. Discuss insights of yield decompositions. Wright 2011 description. Adrian 2017 description. Sack 2008 description. Read KW 2005 explanations for declining TP \(<->\) Episodes of negative TP per country \(<->\) Portfolio inflows \(<->\) Taper tantrum. Correlation w/ inflation, changing pi*: do surveys help w/ this; is high inflation a risk factor. LCCS properties in D-S. Hoffman-Shim-Shin. Financial integration. Russia case discussed in D-S: can the 3-part decomposition shed light?
	\end{todolist}
	
	\item Tables.
	\begin{todolist}
		\item Table: report fit of the model for monthly and daily data in the same table.
		\item Table: correlations of YC components w/ model-free measures.
		\item Table: on-impact responses. Event study for TP, LCCS, yE for AEs and EMs.
	\end{todolist}

	\item Annex.
	\begin{todolist}
		\item Descriptive tables: summary of YCs as in Guimaraes, US MPS.
		\item Figures: LPs for US, LPs for AEs, LPs for EMs regions, LPs for AEs blocks.
	\end{todolist}
	
\end{todolist}

\section{Analysis: Medium Term}
\gototoc
\begin{todolist}
	\item Compare my TPMX against Blake et al. Both TP nom and TPsynt.
	\item Update slides.
\end{todolist}

\section{Analysis: Long Term}
\gototoc
\begin{todolist}
	\item MPS from ECB. Which MPS matter more for EMs, Fed or ECB? Does it varies by region (some countries respond more to ECB than to the Fed)?
	\item Correct US YC for credit risk (Chernov, Schmidt, Song).
	\item Create a structure for US data: Vix, KWUSTP, KWUSYP, SVEY, EPU, FFR, USMPS. In daily2monthly for US extract LCNOM, OIS, FF. Add Guimaraes TP.
	\item Estimate Guimaraes w/ daily data for US and compare against KW.
	\item Use 0.25 and 0.5 Y to test results from forward premium puzzle. Whether it holds at short but not at long maturities (Lingareddy 2008).
	\item ATSM estimated with longer series for AEs.
	\item Replicate analysis in Dahlquist \& Hasseltoft on time-varying bond risk premia.
	\item Repeat analysis using dataset of Wu (competes against GSW) to see how sensitive the results are to using GSW.
	\item Replicate Lloyd.
	\item Check whether the removal of CB governors in TRY had an effect on its TP. Did TP decline in countries that adopted IT during the sample period? Need to know when it was announced (not when IT started).
\end{todolist}

\section{Analysis: Completed}
\gototoc
	\begin{todolist}
	\item[\done] TP regressions. Drivers (monthly): Panel regressions for TP, LCCS, yE for AEs and EMs. Currently, no macro data for AEs.
	\item[\done] Cross-check analyses in Stata using monthly/daily data.
	\begin{todolist}
		\item[\done] Check calculation of path and LSAP shocks. Not exact but highly correlated (Path shocks 0.9995, LSAP shocks 0.9946).
		\item[\done] Correlation b/w yP and svyCBP, yP and 2Y yield (70).
		\item[\done] Correlation b/w yTP and svyTP, yTP and synthetic residual10Yon3M (70). Good against svyTP, not good against residual.
		\item[\done] Correlation b/w yTP and RHO/PHI, yTP/PHI and EPU/Vix.
	\end{todolist}
	\item[\done] Update code for LPs:
	\begin{todolist}
		\item[\done] Correct lag of dependent variable for LPs.
		\item[\done] Update LPs for US for all subsamples. Adjust y-axis. 
		\item[\wontdo] Update individual LPs. Used to decide about the shifting. For US LPs were similar.
		\item[\done] Update LPs for AE/EM same sample as before for comparison.
		\item[\done] Update LPs for AE/EM for all subsamples. Define local for subsamples. There is a change in some LPs due to different samples.
		\item[\wontdo] DK standard errors robust to heteroskedasticity, autocorrelation and cross-sectionally dependent. Number of lags = 4. Last b/c they will take more time. too many values error when running xtscc.
		\item[\wontdo] Update LPs for AE/EM for all subsamples after DK and compare vs pre-DK.
		\item[\done] Create blocks for AEs as EM regions.
		\item[\done] LPs for regions of EM and blocks of AE. Use shocks after replace missing values with zeros to increase the number of observations; when calculating LPs for AE/EM using shocks w/ zeros (no missing) values yields essentially the same LPs.
		\item[\done] Merge codes in lp_spillovers and lp_regressions. Codes renamed starting w/ spov and disaggregated per blocks of tasks.
	\end{todolist}
	\item[\done] Mention characteristics of sample: MYR doesn't have an IT regime but similar characteristics (Pennings et al. 2015).
	\item[\done] Reorient paper based on results.
	\item[\done] Identify why LCNOM is not responding to US MPS. Conclusion: After time shift they respond on impact.
	\item[\done] Individual LPs.
	\begin{todolist}
		\item For AEs 24m: different pattern as in US in NOK, SEK; CAD, GBP similar to US; not as strong as in US in AUD, CHF, DKK, EUR, JPY, NZD.
		\item For AEs 120m: AUD, CAD, CHF, DKK, EUR, GBP, JPY, NOK, NZD,SEK similar to US.
		\item For EMs 24m: different pattern as in US in HUF, IDR, PHP, RUB, ZAR; ILS, PEN similar to US; not as strong as in US in BRL, COP, KRW, MXN, MYR, PLN, THB, TRY.
		\item For EMs 120m: different pattern as in US in HUF, IDR, PHP, PLN, RUB, TRY, ZAR; ILS, KRW, MXN, PEN similar to US; not as strong as in US in BRL, COP, MYR, THB.
	\end{todolist}
	\item[\done] Download EPU indexes for AEs to see how it correlates w/ AE TPnom.
	\item[\done] Generate dta file with new US MPS (excluding Sept 2001 meeting).
	\item[\done] Analysis on spillovers: Responses of EM components to US MPS. (regressions as in Curcuru or panel local projections as in ACDM); lags in response; asymmetric effects on different parts of the YC.
	\item[\done] LPs for nominal yields after accounting for timing. Conclusion: On-impact effect improves (it's now positive). In general, similar patterns as before for AE and EM but smoothed. Ending the sample in 2008 doesn't alter the conclusion.
	\item[\done] LPs for synthetic yields, rho and phi after accounting for timing. Conclusion: On-impact effects improve.
	\item[\done] Schedule: codes (ts_*), figures (ts_plots), tables (ts_tables), export dataset as dta, do-file w/ basic analysis, re-estimate the model w/ monthly data after all changes in codes (RMSE, matsout, i_survey etc.) and update figures-tables-dta-do, estimate the model w/ daily data (assess fit and test against results w/ monthly data), event study analysis.
	\item[\done] Construct panel dataset to analyze results in Stata. Include: rho, CIPdev, svys (CBP, INF, GDP), LCNOM, LCSYNT, decompositions (yQ, yP, tp), macro-financial (US, CBP, STX, FX), EPU. Before dataset, construct a field with yQ, yP, tp for ALL countries (EMs/AEs), ssb for EM w/ surveys, sy for ILS and ZAR, ny for AEs. Same tenors (0.25-10Y).
	\item[\done] Add usyc. Compare response of usyc vs results in Brooks, Kratz and Lustig. Results of usyc across panels should be similar to usyc alone. Explain movements in ysynt due to rho and usyc. Conclusion: Responses of 2Y and 10Y yields to target shock similar to ACDM.
	\item Compare monthly w/ daily fit and decompositions. Very closely related, lowest correlation across all countries is 0.9953 followed by 0.9993.
	\item[\done] Combine LP graphs per tenor-group-shock.
	\item[\done] Compare target and path shocks against GSS (2005).
	\item[\done] In Matlab, remove MP1 shock of September 2001 immediately after reading the file.
	\item[\done] Table of local events. Only insightful for TRY.
	\item[\done] LPs for phi after censoring phi \(>=\) 0.
	\item Regional analysis. Do LPs change depending on the region? It doesn't seem to be large differences across regions.
	\item[\done] Add rho as a dependent variable in LPs.
	\item[\done] ACDM controls. Generate ALL variables (lags, forwards, differences). Delete observations for which mp1 = . . Define a business calendar for FOMC dates. Analyze autocorrelation of US MPS. Controls: past and future shocks. Conclusion: LPs are similar with shocks-only case. Small difference in responses when sample ends in 2015.
	\item[\done] Identify why there are so many missing values when Stata calculates forward. Reason: Weekends. Need to define a business calendar to correctly obtain lags, forwards and differences.
	\item[\done] Use CIs from generated output instead of calculated.
	\item[\done] Remove 9/11 obs in regressions from US MPS. However, it needs to be removed since the shocks are generated.
	\item[\done] Analyze LP codes.
	\item[\done] Include country-specific data for EMs in dataset: STX, surveys, real rates.
	\item[\done] Add an end of quarter dummy.
	\item[\done] Include global activity indexes from Hamilton and Kilian.
	\item[\done] Include EPU US and global in dataset.
	\item[\done] Include path and tp shocks in TT_mps.
	\item[\done] Include FX (expressed in LC per USD) in dataset.
	\item[\wontdo] Include USTP, USYP, in dataset. No longer need the daily changes in the components since now using HF target, path and tp shocks.
	\item[\done] Read US MPS. File of wide windows.
	\item[\done] Create dataset for Stata analysis.
	\item[\done] Clean Analysis folder (rp_*).
	\item[\done] Transform script for daily fit into a function.
	\item[\done] Estimate model for daily data to study the effect of QE announcements and local events. Follow ACM (2013) to estimate and assess fit for daily yields - Use OLS. Prepare a new S structure to store the results - no need for new S.
	\item[\done] Compare PCs from pca to smoothed state xs. They are not equal.
	\item[\done] Add daily datasets to structure S.
	\item[\wontdo] Eliminate yields in month after US end-of-month date. Not necessary, once parameters are estimated, only need yields observed.
	\item[\done] Add 0.25 and 0.5 to matsout.
	\item[\done] Add RHO to daily2monthly. In \textit{VarType} and in the \textit{switch} loop. Check whether in \textit{fields} too.
	\item[\done] Function to calculate RMSE. Not included in estimation* b/c it will be used after estimation w/ daily data.
	\item[\done] Fields in structure for baseline estimations for all countries (ssb for EM, sy for ILS and ZAR, ny for AEs).
	\item[\done] Codes: move all parts in ts_plots that add variables to S into a function that can be called by ts_analysis. Leave only code for plotting in ts_plots.
	\begin{todolist}		
		\item[\done] Update forecast_cbpol.m and append_surveys.m to use new field name (svycbp).
		\item[\done] Split ts_plots into functions: plots: correlations, pca.
		\item[\done] Call all functions from ts_analysis.
		\item[\done] Take functions datesmnmx and syncdtst outside of ts_plots and test.
		\item[\done] Create add_vars function.
		\item[\done] After estimation, add fields to structure S: macro vars, svys, brp, epu, realrt. Needed for plots and tables.
		\item[\done] Save surveys in S as: svyinf, svygdp, svycbp. All maturities. 
		\item[\done] Add policy rates data to macro data in structure of all countries.
	\end{todolist}
	\item[\done] Rename section 4 from Results to Yield Curve Decompositions. Section 5 would then addressed international spillovers.
	\item[\done] Mention explicitly that 3 factors b/c they explain more than 99\% of the variation in yields.
	\item[\done] US factor + non-US factor in each one or only in some of them?
	\item[\done] Download USTP01 USTP10, USYP01, USYP10 from KW, and construct USyP.
	\item[\done] TP correlations of AEs with: USTP, Vix, CIPdev.
	\item[\done] PCA for TPsynt, LCCS and expectations. Common factor, commonalities (TPs, yP, LCCS) all, ST vs LT. Not real rates b/c dates don't match.
	\item[\done] PCA for yields synthetic and nominal. Variability explained by PC1-PC3. 
	\item[\done] Correlations of LCNOM components w/ alternative measures. Similar to Table 1 in Curcuru et al (2018).
	\item[\done] Plot TP vs LCCS, USTP, Vix, EPU, local inflation. Not debt flows from TIC b/c no data yet.
	\item[\done] TP correlations of EMs with: USTP, Vix, EPU index, LCCS, local inflation. Previous odd results w/ correlations might be explained by differences in estimated TPs. Use figure ssb_ny_tp.eps.
	\item[\done] Plot inflation vs 10Y yield. They tend to move together.
	\item[\done] Compare TP of Guimaraes vs KW. Add QE-TT dates. Compare USTP against TP in AEs and EMs.
	\item[\done] Plot TPsynt vs TPsvy. TPsynt follows TPsvy closely.
	\item[\done] Drivers of yields. Compare TPsynt, LCCS and yE against nominal yield to see drivers of EM yields.
	\item[\done] Bond risk premia. Compare TPnom vs TPsynt+LCCS. Construct BRP series. Plot its components. Compare against TPnom.
	\item[\done] Term structure of TP. Averages per country. QE events added.
	\item[\done] function for synchronizing series.
	\item[\done] Compare yEsynt and yEnom (w/ and w/o surveys) to see if they are similar. If so, it supports TPnom = TPsynt+LCCS -> TP and RP are not the same thing for EMs. Hypothesis: yEsynt and yEnom  closer w/ surveys. Won't: Compare when sgmS free vs fixed.
	\item[\done] Feeling of fit. Compare ylds vs yQ. They are very close for most countries.
	\item[\done] Plot LT real interest rates = ssb_yP - INF surveys. Are LT real rates similar? It seems that for most countries it fluctuates b/w 0 and 1 pp.
	\item[\done] Plot yP for EMs and AEs. Is there a trend? No a marked one.
	\item[\done] Plot ssb_yP vs CBP surveys. 
	\item[\done] Compare TPsynt in EMs vs TPnom in AEs.Declining trend more marked for AEs, in EMs TP < 0.
	\item[\done] Compare TPnom vs TPsynt for AEs. How big is the difference? Similar for most. Differences for AUD, NZD, SEK.
	\item[\done] Plot TPnom for AEs. Trends? Declining.
	\item[\done] Analyze TP synt.
	\item[\done] TP for AEs. Reaction to QE events.
	\item[\done] Compare TP different versions: nominal vs synthetic, ylds only vs surveys, sgmS free vs fixed. 
	\item[\done] Plot TP different versions individually. Add key dates when plotting TPs: target adopted/changed, QE announcements, local events (capital controls). 
	\item[\done] Plot: inflation, policy rate, expectations of inflation.
	\item[\done] Plot survey forecasts of inflation. See whether there is a trend. For some countries.
	\item[\done] Read old version of results.
	\item[\done] Review old code in Analysis folder.	
	\item[\done] Estimate ATSM for AEs and EMs. Save TP w/ JSZ and w/ surveys for LCNOM and LCSYNT.
	\item[\done] After estimating ATSM, generate yields (P, Q) and TP for ALL maturities to get balanced panels (output tenors vs input tenors). Only report 1Y, 5Y, 10Y tenors in all cases.
	\item[\done] Make sure all monthly data (yields, surveys, macro) has SAME end-of-month dates.
	\item[\wontdo] Add als in pca to be able to estimate JSZ for initial values. Need balanced panel to use JSZ.
	\item[\done] Generate dataset w/o fitting NS to YCs.
	\item[\done] Replicate Guimaraes but w/o including the survey maturity for 10Y to see how much that maturity helps pinning down the parameters of the model. It helps a lot, especially with shorter samples.
	\item[\done] Explicit that longer series for AEs and surveys for EMs. Also, no surveys for AEs for comparison w/ other studies. % :)
	\item[\done] Decompose US YC.
	\item[\done] Use surveys to decompose US YC and compare to Guimaraes.
	\item[\done] Look for ST expectations in CE for EMs to complement LT ones.
\end{todolist}

\section{Literature}
\gototoc
\begin{todolist}
	\item Serge Jeanneau and Camilo Tovar (2007) on development of LC bonds markets (before short-term debt and dollar-indexed liabilities). Jeanneau and Tovar (2008): for developments in LC bond markets.
	\item Papers suggested by Duffee: Agustin, Chernov, Schmid, Song; Banque de France.
	\item Market segmentation: D'Amico et al., Vayanos. Theories may apply to EMs. Instead of market segmentation on different sectors of the US YC, on different EMs.
	\item Plumbing of the global system. Papers discussed in NBER SI.
	\item BIS bulleting 2020.
	\begin{todolist}
		\item Hoffman Im and Shin (2020) for developments and references on LC bond markets. Emerging market economy exchange rates and local currency bond markets amid the Covid-19 pandemic.
		\item Dollar funding costs during the Covid-19 crisis through the lens of the FX swap market.
		\item Leverage and margin spirals in fixed income markets during the Covid-19 crisis.
		\item Literature on sovereign default (real, nominal, current account behavior) --curvature in utility introduces risk aversion, imperfect financial markets --, currency denomination, original sin.
		\item Risk premia in international macro-finance (Lustig).
	\end{todolist}
	\item Sekkel: C-P regressions in international bond markets, implications for replicating Dahlquist \& Hasseltoft.
	\item R\&R (2010): The Forgotten History of Domestic Debt.
	\item IMF(2019): A Guide to Sovereign Debt Data.
	\item G20(2020): Recent developments in EM LC bond markets.
	\item Erce \& Mallucci (2018): Selective Sovereign Defaults.
	\item Banxico: What determines the neutral rate in an EM.
	\item Gagnon et al for effects of QE on TP.
	\item Caballero and Simsek (2020): A Model of Asset Price Spirals and Aggregate Demand Amplification of a Covid-19 Shock. UMP buys risk when investors become risk intolerant. Asset prices affect real economy w/ a lag (negative output gap w/ a positive asset gap b/c try to boost AD).
	\item Trend Inflation in Advanced Economies (Garnier, Mertens, Nelson). Multivariate version of UCSV from Stock and Watson (2007).
	\item Bostanci, G., and Yilmaz, K. (2020),"How Connected is the Global Sovereign Credit Risk Network?", Journal of Banking and Finance, forthcoming, 2020.
	\item Diebold-Yilmaz (2014). Book (Financial and Macroeconomic Connectedness) is available online in JHU library. Website: http://financialconnectedness.org/index.html.
	\item Nyhom (2000): Regime shifts in the Danish term structure of interest rates.
	\item Ilzetzki and Jin (2020). The puzzling change in the international transmission of US macroeconomic policy shocks. Important: monthly data.
\end{todolist}

\end{document}
