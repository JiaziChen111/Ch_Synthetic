
Importance of decomposition:
- risk management
	- simulations
- asset allocation
	- diversification across investment horizons
- monetary policy analysis
	- understand effects on MP decisions: MP effects on YC components and effects of YC components on financial assets
	- comovement of LC yields w/ foreign yields


EM have increased its issuance of sovereign debt in local currency (DS 2016, Ottonello and Perez 2019)

Default risk for EMs is non-negligible (DS 2018, Galli 2020)

Theoretical explanations for positive default risk in LC (when government lacks commitment to repay and to not inflate)
- Large amounts of private external debt (DS 2016)
- in bad times, incentive to use inflation to smooth public spending increases along with default probability (Galli 2020)
- GDP is more volatile in EMs

Rationalize recent surge in LC borrowing (DS 2016, Engel and Park 2018, Ottonello and Perez 2019)

Sovereign defaults are generally followed by episodes of abnormally high inflation (Na et al. 2018; Reinhart and Rogoff 2009, 2011)

Na et al. (2018) rationalize the joint occurrence of defaults and large currency devaluations: after a default, it is optimal to devalue the currency when wages are sticky to reduce the real value of wages and stimulate the economy.

Galli (2020): 
Inflation and default risks co-move
Default and currency premia are positively correlated
Inflation has a dual purpose, it dilutes the real value of external debt (tax on foreign lenders) and of domestic money (tax on domestic money holdings).
Inflation is a source of fiscal revenues.
When the demand for public goods is sufficiently inelastic, the incentive to use the inflation to smooth public spending (the tax motive) is strong.

International sovereign yields of AEs: Driessen, Melenberg and Nijman (2003), Diebold, Li and Yue (2008), Wright (2011), Jotikasthira, Le and Lundblad (2015), Dahlquist and Hasseltoft (2016), ACDM (2019)

Decomposition of international yields of AEs: Wright (2011), ACDM (2019)

==============================================

Wright (2011)
Global TP has declined since mid 90s.
Decline is associated with a reduction in inflation uncertainty.
Inflation uncertainty is an important driver of term premiums.

Diebold, Li and Yue (2008)
Global yield curve factors linked to global macroeconomic factors such as inflation and real activity.

Curcuru, Kamin, Li \& Rodriguez (2018)
Is effect of CMP < UMP? No, actually CMP >= UMP.
CMP and UMP affect both USYP and USTP.
CMP mainly affects USYP while UMP mainly affects USTP.
FX is more sensitive to yP than to TP, especially post-GFC.
- The effect of TP on FX appears to be post-GFC.
Foreign yields:
- For AEs, changes in USYP and USTP have similar effects.
- For Brazil, Canada and Mexico, effect of USYP is greater than USTP.
Foreign yields became more sensitive to US MP post-GFC, especially to USYP.
UMP does not exert greater international spillovers than CMP.
Effects of USYP exert effects at least as large as USTP, especially post-GFC [why to both?] -> Challenges faced by foreign economies not entirely due to QE.
Policy easings by the Fed lead to substantial declines in German TP and vice versa: portfolio rebalancing and/or expectations of foreign easing.
-----
Decomposition for YCs of Brazil and Mexico provide further insights of the effects?


Adrian, Crump, Durham \& Moench (2019)
* Sovereign yield curves comove strongly since 90s
* Comovement and spillovers more pronounced post-GFC.
- Comovements (covariance, connectedness, tail dependence) in sovereign yields primarily due to TP.
- Global yields react strongly to US target shocks (panel local projections), w/ a substantial lag primarily via a reassessment of global policy rate expectations.
- US ST -> global ST, US TP -> global TP. Consistent w/ Curcuru et al. (2018)
Factor structure of global TP is stronger than ST and has been increasing since GFC.
Connectedness is primarily driven by TP.
ST of global yields strongly responds to US target shocks but w/ considerable delay.
Prominent role of TP in explaining yield variation: common trend in interest rate risk compensation + substantial proportion of yield variation.
Secular decline in TP variance increasingly accounted for a greater proportion of lower yield variance.
- Secular decline in TP due to inflation uncertainty (Wright 2011).
- Decline in TP is not restricted to the largest economies (also in other AEs and EMs).
TP account for more of the variation in yields since mid-90s.
- Due to SR (flight to quality) or LR (perceptions of interest rate risk -> attitudes toward CB commitments to LR goals) shocks to TP?
HF cycles account for more of the variance and covariance in US yields.
- Is it due to ST or TP? HF cycles account for considerably more variation in the YC components. HF share for TP closely follows HF share in yields. LF cycles account for more variation in TP. 
Increase in TP share of lower variance coincides w/ increase in LF power for diminished variation in risk preferences.
Decreasing contribution of ST to variance coincides w/ elevated HF power -> Investors absorbed news about the path of policy rates more quickly over the last two decades.
- Is there a global trend?

A large fraction of the monthly covariation of sovereign yield curves is due to global term premium comovement.
Global YCs have comoved strongly sin late 90s, decline during GFC. Comovement picked up again around 2013 driven by increase in TP correlation in part b/c global TP comove strongly w/ US TP.
Global YC connectedness has been elevated since 2000, decline in GFC.
ST and TP components connected to similar extent until 2007. ST connectedness declined since GFC. TP connectedness declined in GFC and picked up again around 2010 to levels not seen before -> Innovations in TP have become more important for the variation of local sovereign yields.
Tail dependence: sharp movements in one bond market spill over into other bond markets.
Secular decline in tail dependence since 2001. Increase in tail dependence since 2013 mainly driven by TP -> Risk attitudes of global bond investors might be a source of comovement in tail events.
- Tail dependence in yields is higher b/w countries in the same economic region.
- Tail dependence for the TP has increased sharply -> Sudden jumps in TP have become more synchronized across countries.

Increased correlation and synchronized jumps in yields -> diversification has become increasingly difficult.
The component of yields relevant for optimization is expected excess returns, which is related to TP (unobserved).
HF cycle share of TP variation declined recently while share of LF cycle increased -> tighter covariance among expected returns may reflect LF phenomena (LSAP, MP goals: IT), which has implications for investors w/ longer investment horizons (may not be well-diversified).

Propagation of MP shocks
Since comovement, spillover, tail dependence, MP international spillovers?
Response of global yields to US target shocks is strongly increasing w/ horizon. Delayed and protracted response of global yields to US target shocks (more than 1-to-1 w/ a substantial delay). 
Due to expectations of further Fed tightening or repricing of interest rate risk (higher TP)? 
ST increase strongly and persistently in response to a Fed tightening -> Investors expect global CB to follow Fed's decision (not due to FX movements).
Sharp reaction of global yields to US target shocks is driven by a reassessment of global policy rate expectations (ST) not by a repricing of global interest rate risk (TP).
US path and TP shocks lead to substantial adjustments of global yields through TP components.

Alternative lens for the propagation: Betas estimated in the frequency domain help determine extent to which lead-lag relationships persist. 
If shocks are sluggish and gather momentum, LF should have meaningful power to explain covariance b/w front and back end of YC.
- Betas of ST increase from HF to LF consistent w/ shocks being persistent (ST effect increasing w/ horizon) but modest contribution of LF to total variation (considerable HF noise in transmission toward the back of the YC).
- Betas of TP do not increase with frequency consistent w/ transmission from the US policy rate path to revisions in foreign anticipated policy paths (not revisions in TP).
- These mechanisms might not be very stable.


Dahlquist \& Hasseltoft (2016)
Excess return predictability and time-varying risk premia are two sides of the same coin.
Rational variation in risk premia arise from either: time-varyings economic risks or time-varying risk aversion among investors.
Future excess returns can be predictable even if markets are efficient: investors require higher risk premia during economic recessions.
The expectations hypothesis (zero or constant risk premia) is rejected in international data.
- The is significant time variation in international bond risk premia.
CP regressions predict excess returns across countries. 
- Predictive performance deteriorated in GFC. Reasons: UMP, fiscal policy uncertainty, time variation in funding and liquidity conditions, credit risk.
A global forecasting factor predicts local bond excess returns significantly -> commonality in risk premia across international bond markets.
- Global and systematic factors drive changes in the riskiness of nominal bonds.
Global factor is closely related to US bond risk premia -> US risk premia matter for international bond returns.
The dynamics of the global factor are countercyclical and predict future economic growth.
- A rise in global bond risk premia is contemporaneously associated with bad economic times but signals improved times ahead.
Bond risk premia turn negative during some periods.
Negative bond risk premia: nominal bonds act as hedging assets rather than risky assets.
No evidence of state dependency in bond markets: global factor predicts returns across business cycles (not only in recessions).
Nominal bonds are risky assets. Theoretical explanations of time variation in bond risk premia (predictable excess returns): habit formation generate time-varying investors' risk aversion (Buraschi and Jiltsov 2007), time-varying long-run economic risk model (Bansal and Shaliastovich 2013), model in which the severity of disaster risk varies over time (Gabaix 2012). (Piazzesi and Schneider 2009).
Portfolio rebalancing theory: changes in the supply of government debt affect the level and slope of the yield curve (Krishnamurty and Vissing-Jorgensen 2011, 2012). Preferred-habitat theory: increase supply of LT government debt steepens the yield curve and predicts excess returns on LT bonds (Greenwood and Vayanos 2014).


Jotikasthira, Le and Lundblad (2015)
Yield curve fluctuations across different currencies are highly correlated.
Channels through which macroeconomic shocks are transmitted across borders.
A world inflation factor drives risk compensation for long-term bonds.


Driessen, Melenberg and Nijman (2003)
A world level factor explains approximately 50\% of the variation in international bond returns.


Meldrum, Raczko and Spencer (2018)
Joint modelling does not bring any material benefits in capturing the dynamics of bond yields.
Joint models of US and German nominal yields do not offer economically significant advantages in fitting the cross-section of yields or predicting future yields. 
Joint models of US nominal and real yields do not offer economically significant advantages in fitting the cross-section of yields or predicting future yields.

Jeanneau and Tovar (2008): LA's LC bond markets - An Overview
there has been a significant change in the composition of government debt. Currency-linked debt has been phased out in a number of countries, including Brazil and Mexico, as part of debt management programmes aimed at reducing vulnerabilities to external shocks. In addition, the relative share of fixed rate debt has increased in most countries. Progress has been particularly notable in Mexico, where the share of fixed rate securities amounted to about 40\% at the end of 2005, versus less than 5\% in 2000. 
there has been a gradual extension of the maturity structure of government debt in
local currency. This has been achieved in part through a shift from short-term to fixed rate bonds and through a lengthening of the maturity of fixed rate bonds
The wider availability of longer-dated bonds is beginning to provide a useful
representation of the term structure of interest rates. Notwithstanding this progress in the region, the amount of fixed rate securities issued at longer tenors remains in most cases limited, as reflected in the relative stability of the weighted average maturity of new issues.
Secondary market trading in domestic bonds, a common measure of liquidity, has
expanded in recent years, but it remains low relative to mature markets. tightness of the market: efficiency with which market participants can trade. local markets for
fixed rate government securities do not appear to be very tight relative to the US market.
Indeed, bid-ask spreads, which provide an idea of the costs incurred by market participants in executing transactions, are significantly higher in Latin America (3-5bp) than in the United States (0.8-1.6 bp).

Kearns, Schrimpf and Xia (2018)
In support of the bond risk premium spillover channel, financial openness unambiguously emerges as the strongest factor in explaining the extent of the sensitivity of a country's interest rates to monetary policy shocks in major AEs.

Ramos-Francia and Santiago García-Verdú (2015)
the possibility of structural change in the policy rate, exchange rate, and long-term interest rate channels generally depends on the EME in question.
Results highlight the potential for a renewed interdependence between the monetary policies in most EMEs and US monetary policy. They also underscore the importance for most EMEs of taking appropriate policy measures, in particular as the United States begins tightening its monetary stance.
A better understanding of the international transmission of US monetary policy can therefore provide critical assistance to policy planning within EMEs.
For the exchange rate channel, the results depend on the specific EME as well. There is less evidence of possible structural changes in this channel across the battery of tests. Expected: EMEs have, for a relatively long time now, maintained a stationary inflationary
process (Noriega, Capistrán and Ramos-Francia (2013)), they have not monetized their fiscal deficits, ending fiscal dominance. This has produced a much lower pass-through.
The long-run interest rate channel also offers mixed evidence. the secular decrease in
long-term rates in the US that began in the early 1980s accelerated, given
the search-for-yield phenomenon, and was further supported by elements such as
US demographic dynamics and the implementation of UMP. In fact, the fall in the long-term rates in EMEs has been faster than their corresponding individual deflationary processes [indicating external influence].
In the long-run rate channel, other factors have surely played a role as well and thus have led to differing results among EMEs. For instance, some of these economies have implemented capital controls, established macroprudential policies, allowed credit booms, or had more flexibility in their fiscal policies. All in all, it would have been a surprise to observe uniform results for the long-term interest rate channel.

Carrillo,Rocio Elizondo,Rodríguez-Pérez and Roldán-Peña (2017)
Recent evidence shows that, in the last 25 years, the natural (or neutral) rate of interest of some advanced and emerging economies has followed a common downward trend, which suggests that there might be international factors behind the determination of equilibrium real interest rates worldwide. 
in almost all studies r* has followed a downward trend that started well before the global financial crisis, although it strengthen during the crisis.
Among relevant external factors, we find that foreign aggregate shocks, the instrumentation of monetary policy in advanced economies, and the gradual reduction in the global long-term real interest rate may explain recent trends in the Mexican natural rate of interest. 

Borensztein, Philippon and Zettelmeyer (2001)-Monetary Independence in EMs - Does the FX Regime Make a Difference
Early paper on comovement

Pan and Singleton (2008)
document the strong correlation of the VIX with changes in sovereign CDS

Kim and Lim (2016)
- the degree of capital mobility may differ between emerging and advanced countries. When capital mobility is strongly restricted, the changes in the interest rate may not affect capital flows and exchange rate in the standard way. For example, under capital account restrictions on capital inflows, the interest rate rise may not induce capital inflows and appreciate the exchange rate.
Schindler et al. (2015): stronger degrees of capital controls are imposed in emerging countries than in advanced countries.
- the degree of exchange rate flexibility does not explain the differences in the exchange rate responses between advanced and emerging countries.
- we don’t find any direct evidence that foreign exchange intervention plays some role
in generating the exchange rate puzzle in emerging countries.
- When the financial market is less developed, the standard monetary policy transmission mechanism may not work well. For example, when the bond market is less developed, an interest rate increase may not induce capital inflows into the bond market, and have a limited effect on exchange rate. the degree of financial market development does not seem to fully explain the difference in the effects of monetary policy on exchange rate.

Book Singleton (2006)
Zero coupon defaultable bond seen as a portfolio of two securities: (1) security paying \$1 contingent on survival at maturity, (2) security that pays a recovery at default if it occurs before maturity
Reduced form models: price using a default-adjusted discount rate
Structural models: price of riskless zero coupon bond minus the value of a put option on the value of the firm
Only the timing of default is priced, assumes that recovery risk is not priced which would introduce a risk premium.
Market price of risk \(\Lambda\) of the j-th risk factor can be thought of as the price (per unit of volatility) of shocks to the j-th  risk factor.
\(\Lambda\) is the vector of risk premiums (volatility in the price induced by volatility in the risk factors) required for each unit of volatility of the N risk factors.
\(\Lambda\) is independent of the cash flow pattern of the security being priced and so is common to all securities with payoffs that are functions of the risk factors Y.

Pan \& Singleton (2008)

Hu, Pan \& Wang (2013)
examine the informativeness of noise in the price of U.S. Treasuries as a measure of overall market illiquidity.
the U.S. Treasury market is one of the most active and liquid markets, one with the highest credit quality, and thus is the number one safe haven during crisis.
daily aggregate of cross-sectional pricing errors
our noise measure comes from the U.S. Treasury bond market—the market with the highest credit and liquidity quality, and the number one safe haven during crises
rather than being a measure specific to the Treasury market, our noise measure is a reflection of overall market conditions
liquidity risk is indeed priced by hedge fund returns
high exposure to market-wide liquidity risk is a key driver for currency carry profits


Kolasa \& Wesolowski (2020)
international aspect of asset market segmentation plays a key role, allows our model to account for substantial comovement in the term premia, direction and size of capital flows

we introduce market segmentation between short- and long-term bonds
two dimensions of asset segmentation in our model: across the term structure and across borders. The former, i.e. excluding restricted households from short-term bond markets and making unrestricted ones subject to transaction costs whenever they adjust their positions in long-term bonds, is borrowed from the previous literature and its role has been extensively discussed by Chen et al. (2012). These assumptions limit the arbitrage between short-term and long-term bonds issued in a given currency, resulting in fluctuations in the term premia that have effects on real activity. Absent transaction costs, short and long-term bonds would become perfect substitutes under certainty equivalence, while allowing restricted households to trade short-term bonds would effectively eliminate transaction costs through arbitrage.

The key assumption we make is that households cannot trade short-term bonds issued abroad.8 This restriction is in line with the data for small open economies, indicating that short-term debt securities account only for a tiny proportion of international financial flows to these countries. In particular, according to the Coordinated Portfolio Investment Survey held by the IMF, the share of short-term instruments in this category is very small and stable (the median value moving between 2.2\% and 5.2\% over the period 2006–2018).
[In line w/ my findings that connectedness for 3M yields is small]

Our market segmentation has important implications for the model dynamics.
transaction costs associated with trade in long-term bonds, which can be interpreted as an endogenous risk premium.
if all transaction costs were equal to each other, as it would be the case if we allowed for trade in short-term bonds, we would obtain the standard UIP condition without any risk premium.
shocks affecting the composition of the large economy's bond supply imply very tight comovement between the term premia.

our model is able to account for the observed increase in the cross-country term premia comovement during the period of quantitative easing compared to the per-crisis times.

fluctuations in the term premium essentially reflect the current and planned portfolio rebalancing decisions made by agents


Obstfeld (2015)
declining reliance on cross-border bank lending reflects domestic financial deepening
growth in bond finance after the GFC
increasing EME recourse to nonbank funding sources has created new exposures
**the remaining external and domestic debt is increasingly denominated in domestic currency (Lane and Shambaugh (2010), Miyajima et al (2012), Turner (2012))
more reliably moderate inflation itself has helped to promote LC denomination liabilities.
One of the most potent channels for international monetary and financial
transmission clearly runs through long-term interest rates.

Miyajima, Mohanty, and Chan (2012), and Turner (2013)
Yields on EM debt securities in LC have fallen in tandem with those of advanced economies and have shown increasing tendency to move in sync with those of advanced economy bonds

McCauley (2017)
monetary easing induces FX depreciation, and implies FX appreciation and disinflation for ROW.
lower bond yields in a key currency such as the USD tend to lower LC bond yields around the world. Again, the influence of domestic MP in the form of setting short-term interest rates is reduced.
Resistance to FX appreciation has also largely driven the reserve accumulation.
Evidence that the TP is subject to a strong global influence -> CB operations in bond markets can have effects that spill over through the global bond market
If CB can influence TP by LSAP, there is much scope for bond market interaction
LSAP are intended to compress the TP in the UST bond market (Bernanke 2013): LSAP works through prices (compressing the TP). Interpretation received support from Gagnon and others (2011), Krishnamurthy and Vissing-Jorgensen (2011), and D’Amico and King (2013).
The stock of USD bonds issued by non-financial borrowers outside the US proved most responsive to the Fed’s compression of the TP
LSAP may not have been intended to change the behavior of borrowers outside
the US; nevertheless, it induced them to build up their USD liabilities. 
Lower yields for UST bonds lead to lower yields on other governments’ bonds, as long as global investors balance their bond portfolios across markets that are open to
investment (IMF 2014).
Bauer and Neely (2014) find that the effect worked through shared TP (that is, through channels other than correlated expectations of future short-term rates) in the larger German and Japanese markets. Rogers, Scotti, and Wright (2014) confirm Neely’s findings using high frequency futures data.
Obstfeld (2015) reports LT level (cointegration) regressions that suggest that major government bond markets move in sync with the UST market, with a median half-life of adjustment of about one year
During Fed tightening cycle in February 1994, US and European monetary policy diverged, but US and European bond yields tracked each other higher, with the TP widening in Europe. CBs were not playing follow-the-leader, but bond market investors were. 
EM reserve managers’ purchases of bonds in reserve currencies could only have reinforced the effect of QE in lowering bond yields in the key currencies, especially the USD
It is a plausible conjecture that 2014 saw “reverse causation” in trans-Atlantic bond markets; that is, the euro area bond market led and the US bond market followed. This conjecture raises the broader question of policy frictions in the global bond market.
EM CB sales of UST could accelerate and increase the dislocation of yields. 

Bernanke (2013)
In the LT, real interest rates are determined primarily by nonmonetary factors, which in turn is closely related to the underlying strength of the economy. The fact that market yields currently incorporate an expectation of very low short-term real interest rates over the next 10 years suggests that market participants anticipate persistently slow growth and, consequently, low real returns to investment. In other words, the low level of expected real short rates may reflect not only investor expectations for a slow cyclical recovery but also some downgrading of longer-term growth prospects

Turner (2014)
* Movements in US LT interest rates can have major implications for both monetary policy and financial stability in EMEs.
*The global LT interest rate now matters much more for the MP choices facing EMs than a decade ago. The low or negative TP in the YC in the advanced economies from mid-2010 has pushed international investors into EM LC bond markets: by lowering local LT rates, this has considerably eased monetary conditions in the EM.
The development of a market-driven long-term interest rate has far-reaching implications for both monetary policy and financial stability. 
By borrowing in their own currency, governments avoided the currency mismatch risks created by heavy foreign currency borrowing in previous decades. 
EM corporates borrowing abroad to invest in domestic assets creates currency mismatches.
When extremely easy external financing conditions allow  EM corporations to borrow cheaply from abroad, local banks have to look for other customers – so that domestic lending conditions facing most local borrowers actually ease more than the expansion in total domestic bank credit aggregates suggest. A tightening in external financing conditions would reverse this … small firms might then find it harder to get finance
even if total domestic bank credit continues to rise. 
Chung et al (2014): the deposits of non-financial corporations are more procyclical than other bank deposits.
The deeper integration of EMEs into global debt markets has made EM bond
markets more sensitive to bond market developments in the AEs. A crucial change has been a transformation of LC debt markets in EMEs over the past decade or so. The proportion of government debt denominated in LC now dwarfs that of denominated in FC.
LC bond markets have, then, become much larger. They have also grown longer in maturity, and they are now, most important, closely integrated with
global bond markets. Foreign holdings of EM LC bonds have risen.
Miyajima et al (2012): EM LC bond yields have moved closely with US yields since 2005.
The impact on long-term LC government bond yields in EMEs of the decline in the term premium in 10-year US Treasuries has been remarkable. The borrowing costs of EM governments have therefore been greatly reduced.
McCauley et al (2014): the TP compression in US Treasuries has significantly stimulated offshore dollar credit via bond markets post-GFC.
The 5-year forward 10-year yield on US Treasuries should be free of changes in expectations about near-term short-term rates.
It has long been recognised that monetary conditions in an open economy change
not only when the ST policy rate changes, but also when the FX changes. 
* The new factor in many EMEs over the past decade is the greater importance of
domestic long-term interest rate. This is strongly influenced by long-term interest
rates in the main financial centres due to LSAP to lower the LT interest rate, and so stimulate AD.
MP stance in many EMEs has been much looser before May 2013 than looking at just the policy rate would suggest because of the substantial fall in real LT rates. 
The policy rate and the LT rate affect different components of real GDP. 
Monetary conditions could be characterised along at least 3 dimensions: ST policy rate, FX, LT rate. Implications: (1) quantification of the stance of MP must consider all 3 variables, (ii) the stance of MP becomes more uncertain, (iii) MP independence is weakened.
Assuming the country’s credit standing is constant, the LT rate in the LC will be heavily influenced by developments in USD bond markets. There is a loss of independence irrespective of the country’s choice of FX regime. This conclusion is not new. Calvo and others demonstrated in the early 1990s the importance of US interest rates for EM capital flows (Calvo, 2013). In the 1980s and early 1990s, it was ST USD
rates that dominated b/c international bank lending was a key component of capital flows. Since the recent crisis, however, capital flows via bond markets have become more important, making LT USD rates crucial (McCauley et al, 2014). 
The LT interest rate is also fundamental for financial stability. In the absence of sovereign default risk, provides the basic discount rate, and is thus central to the pricing of all LT assets.
The terms on which financial intermediaries provide maturity transformation will influence the TP.
CBs in EMEs have to think very carefully about the size of the TP in the YC for their own government bonds.
Foreign investors holding EM bonds have been reminded (after LC depreciations due to TT) how exposed they are when global financial conditions change – even when the economic circumstances in the EME itself remain constant.


Miyajima, Mohanty, and Yetman (2014), and Chen and others (2015) shows a tighter linkage bond markets in emerging markets now respond more to changes in global bond markets than they did a decade ago.
The integration measured by Obstfeld (2015) can easily be read as the influence of the largest, deepest, and most liquid government market on other bond markets.

Miyajima, Mohanty \& Chan (2015)
what factors dictate the yield on EM local currency government bonds?
Foreign investors' appetite was particularly strong for emerging market (EM) local currency denominated bonds.
cumulative net inflows to mutual funds dedicated to EM bonds grew from a little over \$20 billion at the end of December 2009 to a peak of \$180 billion at the end of May 2013. Around half of those inflows were directed to local currency-denominated bonds. During this period of relatively high risk aversion, the yields on EM local currency government bonds typically fell and tracked closely the US Treasury yield, widely accepted as the “risk-free” global benchmark yield.
Historically, increased risk aversion and tighter global monetary conditions tended to reduce capital flows into EM debt products.
increase in the foreign ownership of EM local currency bonds in the past decade.
“original sin” hypothesis: EMEs lack the capacity to borrow in their own currencies because international transactions are mostly denominated in currencies
What drives foreign investors' appetite for EM local currency government bonds?
- foreign investors' demand for EM local currency bonds has risen because institutional investors have diversified their portfolio
- global factors are key in affecting foreign investors' demand for EM local currency bonds -> global factors, rather than local factors, determine the return on EM local currency bonds and foreign investors' interest
- Diversification benefits depend crucially on the behaviour of the exchange rate. Because EM currencies tend to be more volatile than those of advanced economies, any diversification benefits from EM assets are likely to be small.
to be attractive, returns on EM bonds should be determined more by domestic factors than global ones, and they must also be resilient to various shocks
To the extent that yields on EME LC bonds are determined by domestic growth and monetary conditions in EMEs, they could offer the much needed diversification opportunity to investors.
Turner (2014) argues that domestic monetary conditions in EMEs are increasingly dictated by global factors through their impact on domestic bond yields.
there is very little systematic evidence to date regarding the determinants of EM local currency bond yields.
Jaramillo and Weber (2013): Fiscal variables are thus more important determinants of EM domestic bond yields during times of stress than in tranquil periods.
Piljak (2013) argues that domestic macroeconomic factors, particularly monetary policy and inflation, are more important than global factors for domestic government bond markets in EMEs.
Mohanty (2014): US monetary policy has had a significant effect on domestic bond yields in EMEs since 2009. term premia seems to be particularly more sensitive to global events.
Adam et al. (2014): As the share of domestic non-bank financial institutions' holding increased, domestic government bond yields became less sensitive to movement in global bond yields during times of extreme market volatility.
Results: 
LC government yields in EMEs are determined by domestic factors.
US long-term interest rates became a more significant determinant of EM LC government yields.


Gilchrist, Yue \& Zakrajsek
intense interest in international monetary policy spillovers.
using high-frequency price data on USD denominated sovereign bonds, we empirically quantify the transmission of U.S. MP in international bond markets. By focusing on dollar-denominated sovereign bonds, we are able to analyze how U.S. monetary policy affects sovereign yields and spreads, a question that is more difficult to address using bonds denominated in LC, an asset class for which policy-induced fluctuations in exchange rates are a direct complicating factor
U.S. monetary policy is transmitted very effectively to both shorter and longer duration yields on USD denominated sovereign debt. The spillover effects of conventional U.S. MP across the sovereign bond portfolios of different durations are much more uniform compared with the unconventional policy regime
The second set of empirical exercises focuses on sovereign default risk. quantify more accurately how U.S. monetary policy affects sovereign default risk across the conventional and unconventional policy regimes.
* Consistent w/ what I found:
- countries with a speculative-grade rating: conventional U.S. monetary policy actions have an economically large and statistically significant effect on credit spreads of dollar-denominated bonds of countries with a speculative-grade credit rating. credit spreads on risky sovereign debt are estimated to narrow (widen) significantly in response to an unanticipated U.S. policy easing (tightening) during the conventional regime. an unanticipated tightening of U.S. monetary policy during the conventional regime widens credit spreads on risky sovereign debt directly through the financial channel (nor due to FX depreciation).
- sovereign bond yields for low risk countries are estimated to decline (increase) by about as much as the yields on comparable U.S. Treasuries in response to a conventional U.S. monetary policy easing (tightening). conventional U.S. monetary policy affects macro-economic and financial conditions in low-risk countries primarily through its impact on bilateral exchange rates.
- the unconventional policy actions undertaken by the FOMC between the end of 2008
and the end of 2015 did not systematically affect the level of sovereign credit spreads
across the credit quality spectrum, despite the fact that those actions had economically large effects on the bilateral exchange rates of both low- and high-risk countries
Bowman, Londono, and Sapriza (2015): empirically quantifies the spillover effects of U.S. unconventional policies on EMs. U.S. unconventional monetary policy
measures induced a significant portfolio reallocation among investors and led to a
notable repricing of risk in global financial markets.
Albagli et al. (2018): U.S. monetary policy spillovers to LT foreign bond yields have increased substantially after the global financial crisis. 
Gagnon et al. (2017): Fed's UMP-prompted increases in U.S. bond yields are associated with increases in local
currency foreign bond yields and equity prices, as well as with a depreciation of
foreign currencies.
* potentially illiquid nature of sovereign bonds, which would lead to a delayed yield response to U.S. monetary policy announcements so \(h\) up to 6 days.


Bowman et al. [2014]:
- Study the transmission of UMP on exchange rates, sovereign yields and stock prices for a set of 17 emerging economies, and how these effects depend on country-specific characteristics.
- Federal Reserve announcements resulted in sudden upswings in volatility.
- every time an announcement drove down U.S. sovereign yields, the same occurred in EMs. EMs with weaker fundamentals were more vulnerable to U.S. monetary policy shocks.
- the effect of U.S. MPS is significant, especially for local-currency sovereign yields, in many countries, but the magnitude and the persistence of the effect varies tremendously across countries.
- while the Fed's unconventional policies have had an impact on the EMEs, this impact has not necessarily been unusually different from the typical impact
- Hausman and Wongswan (2011) looked at the effects of FOMC announcements from 1994-2005 on financial markets in 49 countries, finding that the cross-country variation in the effects of U.S. monetary policy is largely explained by the exchange rate regime. Bowman et al. [2014] add to the evidence by comparing the effect of standard and unconventional monetary policies.
%- most timely and clear way to observe the effect of MP on the economy is by looking at financial markets.
- Results from event study: EME yields in local currencies tend to experience large fluctuations around U.S. unconventional monetary policy announcements. However, while we see some large moves in exchange rates and stock prices for some countries around some announcements, the results are less uniform and often less statistically significant.

Rogers, Scotti \& Wright (2014)
UMP policies are indeed effective in easing broad financial conditions – not just lowering government bond yields – when policy rates are stuck at the ZLB.
This appears to work largely by reducing term premia, and indeed driving them negative
The pass-through from bond yields into other asset prices generally seems to be bigger for the US than for other countries.
the effects of US MPS on non-US yields are larger than the other way round.
LSAP announcements generally seem to have more pass-through into asset prices other than government bond yields.
the effects wear off, although slowly. That implies that it is not just the stock of central bank purchases of securities that matters for financial conditions, but
also the flow.


Bernanke (2018)
I argue that monetary and exchange-rate policies should focus on macroeconomic objectives, with the problem of spillovers being tackled by regulatory and macroprudential measures, possibly including targeted capital controls, and through careful sequencing of market reforms. 
EMs in particular are subject to powerful gross capital inflows during “risk-on” periods of low market volatility and low risk premiums, and to corresponding sharp outflows during “risk-off” periods. 
Chen et al. (2014) find that unconventional monetary policy had relatively stronger effects, while Bowman et al. (2014) and Rogers et al. (2014) don’t find a significant difference between the effects of CMP and UMP. 
What is the economic mechanism that generates the global financial cycle? financial
cycle is at least in part the product of the behavior of financial intermediaries such as investment banks (Adrian and Shin, 2014; Bruno and Shin, 2015).
When the Fed eases (say), global intermediaries and other investors are reassured about the economic outlook. Consequently, volatility falls and risk appetite increases, leading in turn to higher leverage and rapid expansion of credit. When the Fed eventually tightens, in this story, the process is reversed, possibly violently. 
shocks confined entirely to one country would have global financial effects, as investors rebalance portfolios and hedge the new configuration of risks.
the finding of a global financial cycle does not itself tell us much about their empirical relevance. 
modeling approach: the financial spillovers of monetary policy operate through variations in risk and liquidity premiums, with time-varying risk premium, exogenous to EME policymakers.
The literature on the “risk-taking channel” of monetary policy find that monetary policy actions works in part by affecting investors’ risk preferences, although I am not aware of much theoretical analysis of the channel (Bruno and Shin, 2015). 
flexible exchange rate provides an important extra degree of freedom for policymakers, still holds when monetary policy works through the propensity for risk-taking or other non-standard channels.
Strengthened financial regulation is particularly critical for reducing financial stability
spillovers. That conclusion is in fact an implication of Shin’s work, which attributes spillovers to the risk-taking behavior of international financial institutions.
Notwithstanding the critical importance of maintaining
financial stability, I don’t at this point see a very strong case for diverting monetary policy from the pursuit of its macroeconomic objectives.
Under most circumstances, monetary policy is just too blunt a tool for addressing financial stability risks, including international risks. More-targeted policies, such as financial regulation, should accordingly be the first line of defense under most circumstances.
the best way for the United States and other advanced economies to address the concerns of emerging markets is through enhanced cooperation in regulation, supervision, and financial-market reform. 


Hausman \& Wongswan (2011):
- U.S. monetary policy surprises do influence foreign asset prices. The response varies greatly across countries and assets.
- Global equity indexes respond mainly to the target surprise, exchange rates and long-term interest rates respond mainly to the path surprise, and short-term interest rates respond to both surprises.
- Foreign equity indexes respond mainly to the target surprise (25bp target cut, up 1\%; 25bp path cut, up 25bp), this is consistent with results for U.S. equity markets (Gurkaynak et al., 2005). Exchange rates respond mainly to the path surprise (25bp path cut, dollar down 50bp). Since the path surprise is more related to the term-structure of interest rates, exchange rates are more affected by the term-structure of interest rates than they are by short-term interest rate. Short-term interest rates respond to both the target and path surprises, while long-term interest rates respond only to the path surprise (25bp target cut, down 5 and 3 bp; 25bp path cut, down 5 and 8 bp). U.S. monetary policy surprises have the highest explanatory power for foreign long-term interest rates (linked to the global business cycle) and the least for foreign short-term interest rates (country-specific factors).
- Responses are economically significant since they represent asset price movements over a one-day horizon.
- Mexico's equity index responds almost the same as that from the US. Most currencies respond mainly to the path surprise but Mexico's responds to the target surprise.
- most currencies in developed countries respond significantly to FOMC announcements, while currencies in emerging market countries respond much less.
- The influence of FOMC announcements (R2) is lower in the foreign exchange market than in the equity market. Interest rates are the most affected.
- Asymmetry 1: a path surprise that occurs on a day when there is a change in the target fed funds rate having a larger influence on foreign asset prices than a path surprise on a day with no change in the target rate. asymmetry for the target surprise’s influence on short- and long-term interest rates.
- Asymmetry 2: In equity and foreign exchange markets, the reaction to FOMC announcements on intermeeting days is larger than that on scheduled days. In contrast, long-term interest rates react more on scheduled days.
- Why heterogeneous response? (1) an equity market in a country that has more real and financial integration, a less flexible exchange rate regime, and a larger equity market relative to GDP responds more to U.S. monetary policy announcements; (2) in a country that has a less flexible exchange rate regime, the exchange rate responds less; (3) a country that has a higher degree of real and financial integration with the United States responds more to FOMC announcements, and a country that has a less flexible exchange rate regime also responds more
- financial integration is more important than real economic integration. For EMs, capital controls do insulate countries from foreign monetary shocks. results are consistent with a role for portfolio rebalancing in transmitting U.S. monetary policy surprises. 
- a country that has a more flexible exchange rate regime responds more to the path surprise.
- Interest rates in a country with more real integration with the United States and a less flexible exchange rate regime respond more to FOMC announcements. global long-term interest rates are linked to the global business cycle (in which the U.S. economy plays an important role) and, therefore, do not have that much of a country-specific component.
- the exchange rate regime is an important determinant of how a foreign country’s financial assets respond to FOMC announcements. In addition, real (for short-term interest rates) and financial (for equities) linkages with the United States explain some of the cross-country variation in the response.


Chen et al. (2014) -> cite for effect of US target and path factors on flows
- Analyze effects of MPS and factors
%- MPS: difference between the yield of the next expiring futures on Federal Funds, taken just before an FOMC announcement, and the target Federal Funds rate actually announced. changes in asset prices and capital flows in the 2-day windows following each FOMC announcement given non-overlapping market opening hours around the world.
- U.S. MPS do affect portfolio inflows and asset price movements in EMs; spillover effects were stronger during UMP (especially large during TT); countries with stronger fundamentals are subject to smaller spillovers, especially during the period of UMP.
- Negative (loosening) surprises are associated with increases in equity prices in EMs, lower bond yields, and stronger currencies, also correlated with portfolio flows into equities, while bond flows remain insignificant.
%- event study methodology to isolate the impact and direct effects of U.S. monetary policy shocks on asset prices and capital flows in EMs. Event studies offer a simple identification strategy. A shortcoming of event studies is that they are not able to capture persistence effects.
%- an announcement that tightens the monetary policy stance, could generate a loosening shock if markets had anticipated a larger tightening.
%- If an announcement is fully anticipated, it will be priced in. to correctly estimate the effects of monetary policy, it is imperative to measure the surprise component of announcements, and not just to use dummy variables to isolate event (announcement) days. we find that results differ markedly if we do not control for the sign and magnitudes of monetary policy shocks.
%- announcements can surprise along other dimensions (GSS 2005). Announcement affect longer term rates—more relevant for economic decisions—through the signals they provide about future policy intentions through e.g. statements, forecasts. The path factor seems to account for most of the effects of MP, it is highly correlated with a variety of asset prices.
%- The signaling channel continued to play an important role, but over longer horizons than before.
%- extracting two factors from changes in U.S. bond yields across the yield curve from 1 year to 30-year maturities
%- UMP primarily conveyed information affecting longer-term bonds and term premia. UMP-P surprises, as expected, have a negative skew, since announcements aimed to loosen monetary conditions, sometimes aggressively.
%- signal surprises are consistently larger and more statistically significant than those of market surprises. 
%- Because factors are orthogonal and normalized, the size of coefficients can be compared.
- surprise measures continue to be significant even after controlling for the VIX. The VIX must therefore be distinguished from U.S. monetary policy shocks. the VIX does explain the greater spillovers during the UMP-T phase
- spillovers from U.S MP to the rest of the world are mostly a recent phenomenon, having remained subdued before the GFC.
- “path” shocks generate larger spillovers (Chen et al. 2014).


Rogers et al. (2015) -> cite for effect of US MP on USD
%- MPS: change in five-year Treasury futures from 15 minutes before the time of FOMC announcements to 1 hour 45 minutes afterwards on the days of FOMC announcements.
%- surprises as external instrument that achieves identification without having to use implausible short-run restrictions.
- By affecting exchange rates and foreign interest rates, monetary policy shifts are a
potential source of unintended spillovers onto other countries. the answers are potentially different at the ZLB.
- Several papers achieve identification by positing a recursive ordering in which it is assumed that U.S. MPS have no immediate effect on foreign interest rates. However, there is considerable evidence from the event-study literature, which we reinforce below, showing that global interest rates and exchange rates respond immediately and substantively to U.S. monetary policy shocks.
Existing event study work has shown a strong contemporaneous relationship between
unconventional U.S. monetary policy surprises, U.S. and foreign interest rates, and
exchange rates (see, for example Rogers et al. (2014)).
- we contribute to the event study literature by examining the high-frequency response of currency carry trade excess returns to U.S. monetary policy surprises, with a focus on the ZLB period. 
%- We use a variant of the method of external instruments where the ordering of the variables does not matter in identification. The ordering of variables is irrelevant as a Choleski decomposition will not be used for identification.
%- Because our main objective is to examine event study evidence on the relationship between excess returns and U.S. monetary policy surprises, timing precision is crucial.
%- two-day window leaves the effect of the pure monetary policy surprise confounded with other factors in influencing interest rates and exchange rates on a daily basis.
- Findings: U.S. monetary policy easing shocks lower domestic and foreign bond risk premia, lead to dollar depreciation and lower foreign exchange risk premia.
- Details: expansionary U.S. MPS cause the dollar to depreciate significantly (effect is significantly positive for a few quarters). MP spillovers are greatest for longer term interest rates (significantly negative effect on ten-year interest rates at short horizons, 10 basis points for the UK and Germany and 5 bp for Japan for a 25 bp cut by the Fed); the effect on foreign long bond yields is estimated to be largely due to term premia. Effects of MPS over the pre-ZLB era generally similar but with differences (larger effect on 3M yields and lower in 10Y yields); the evidence that U.S. MPS affect foreign term premia seems a little weaker over the pre-ZLB sample.
-  QE effects vary depending on the type of policy action.


Albagli et al. (2019)


Kolasa \& Wesołowski (2020)
QE generate a much larger inflow of non-residents into LC LT bond markets of EM, resulting in a much sharper appreciation of their real exchange rates.

the small economy's CB by using only CMP can easily control the short end of the domestic YC, but is less powerful in affecting its long end. Therefore, following QE in the large economy that strongly depresses foreign long-term rates, the equalization of ex ante returns on home and foreign bonds is achieved mainly by FX and TP adjustment

Research questions:
Do effects on EM yields is similar to effect on US yields (yP, TP)? EM response is larger
How credit risk alter the effects of risk spillovers in EMs? Not much
How risk spillovers work along the TS? Literature so far focus on just 1 maturity.
What explains the responses of yields? Is the effect on EM yields due to yP or TP or CR? yP and TP over time. Role of CR is to allow to estimate a pure TP.
Do changes in ST and TP have similar effects on EM yields? No
Do risk spillovers story rules out effect on yP?
Which component is more sensitive to global factors (TP) and which to domestic factors (LCCS)?
Do spillover vary by type of MP shock? Yes
Are there cross effects b/w components? Yes
Obstfeld (2015): What factors are most important in determining correlations of LT rates – expected short-term rates, term premia or currency risk premia? According to DY index: TP > YP > CR.
And what are the implications for domestic monetary control?
Bernanke (2018):  documenting the channels through which the putative spillovers operate.

Contributions:
- HFI US MPS.
- 3 types of shocks: effects on YC and components, pre- \& post-GFC.
- decomposition accounting for CR -> clean and robust TP.
- LPs w/ DK SE.
- TS DY index.