
Importance of decomposition:
- risk management
	- simulations
- asset allocation
	- diversification across investment horizons
- monetary policy analysis
	- understand effects on MP decisions: MP effects on YC components and effects of YC components on financial assets
	- comovement of LC yields w/ foreign yields


EM have increased its issuance of sovereign debt in local currency (DS 2016, Ottonello and Perez 2019)

Default risk for EMs is non-negligible (DS 2018, Galli 2020)

Theoretical explanations for positive default risk in LC (when government lacks commitment to repay and to not inflate)
- Large amounts of private external debt (DS 2016)
- in bad times, incentive to use inflation to smooth public spending increases along with default probability (Galli 2020)
- GDP is more volatile in EMs

Rationalize recent surge in LC borrowing (DS 2016, Engel and Park 2018, Ottonello and Perez 2019)

Sovereign defaults are generally followed by episodes of abnormally high inflation (Na et al. 2018; Reinhart and Rogoff 2009, 2011)

Na et al. (2018) rationalize the joint occurrence of defaults and large currency devaluations: after a default, it is optimal to devalue the currency when wages are sticky to reduce the real value of wages and stimulate the economy.

Galli (2020): 
Inflation and default risks co-move
Default and currency premia are positively correlated
Inflation has a dual purpose, it dilutes the real value of external debt (tax on foreign lenders) and of domestic money (tax on domestic money holdings).
Inflation is a source of fiscal revenues.
When the demand for public goods is sufficiently inelastic, the incentive to use the inflation to smooth public spending (the tax motive) is strong.

International sovereign yields of AEs: Driessen, Melenberg and Nijman (2003), Diebold, Li and Yue (2008), Wright (2011), Jotikasthira, Le and Lundblad (2015), Dahlquist and Hasseltoft (2016), ACDM (2019)

Decomposition of international yields of AEs: Wright (2011), ACDM (2019)

==============================================

Wright (2011)
Global TP has declined since mid 90s.
Decline is associated with a reduction in inflation uncertainty.
Inflation uncertainty is an important driver of term premiums.

Diebold, Li and Yue (2008)
Global yield curve factors linked to global macroeconomic factors such as inflation and real activity.

Curcuru, Kamin, Li \& Rodriguez (2018)
FX is more sensitive to ST than to TP, especially post-GFC.
- The effect of TP on FX appears to be post-GFC.
Foreign yields:
- For AEs, changes in ST and TP have similar effects.
- For Brazil, Canada and Mexico, effect of ST is greater than TP.
Foreign yields became more sensitive to US MP post-GFC, especially to ST.
UMP does not exert greater international spillovers than CMP.
CMP and UMP affect both ST and TP, but CMP mainly affects ST while UMP mainly affects TP.
Effects of ST exert effects at least as large as TP, especially post-GFC [why to both?] -> Challenges faced by foreign economies not entirely due to QE.
Policy easings by the Fed lead to substantial declines in German TP and vice versa: portfolio rebalancing and/or expectations of foreign easing.
-----
Decomposition for YCs of Brazil and Mexico provide further insights of the effects?


Adrian, Crump, Durham \& Moench (2019)
* Sovereign yield curves comove strongly since 90s
* Comovement and spillovers more pronounced post-GFC.
- Comovements (covariance, connectedness, tail dependence) in sovereign yields primarily due to TP.
- Global yields react strongly to US target shocks (panel local projections), w/ a substantial lag primarily via a reassessment of global policy rate expectations.
- US ST -> global ST, US TP -> global TP. Consistent w/ Curcuru et al. (2018)
Factor structure of global TP is stronger than ST and has been increasing since GFC.
Connectedness is primarily driven by TP.
ST of global yields strongly responds to US target shocks but w/ considerable delay.
Prominent role of TP in explaining yield variation: common trend in interest rate risk compensation + substantial proportion of yield variation.
Secular decline in TP variance increasingly accounted for a greater proportion of lower yield variance.
- Secular decline in TP due to inflation uncertainty (Wright 2011).
- Decline in TP is not restricted to the largest economies (also in other AEs and EMs).
TP account for more of the variation in yields since mid-90s.
- Due to SR (flight to quality) or LR (perceptions of interest rate risk -> attitudes toward CB commitments to LR goals) shocks to TP?
HF cycles account for more of the variance and covariance in US yields.
- Is it due to ST or TP? HF cycles account for considerably more variation in the YC components. HF share for TP closely follows HF share in yields. LF cycles account for more variation in TP. 
Increase in TP share of lower variance coincides w/ increase in LF power for diminished variation in risk preferences.
Decreasing contribution of ST to variance coincides w/ elevated HF power -> Investors absorbed news about the path of policy rates more quickly over the last two decades.
- Is there a global trend?

A large fraction of the monthly covariation of sovereign yield curves is due to global term premium comovement.
Global YCs have comoved strongly sin late 90s, decline during GFC. Comovement picked up again around 2013 driven by increase in TP correlation in part b/c global TP comove strongly w/ US TP.
Global YC connectedness has been elevated since 2000, decline in GFC.
ST and TP components connected to similar extent until 2007. ST connectedness declined since GFC. TP connectedness declined in GFC and picked up again around 2010 to levels not seen before -> Innovations in TP have become more important for the variation of local sovereign yields.
Tail dependence: sharp movements in one bond market spill over into other bond markets.
Secular decline in tail dependence since 2001. Increase in tail dependence since 2013 mainly driven by TP -> Risk attitudes of global bond investors might be a source of comovement in tail events.
- Tail dependence in yields is higher b/w countries in the same economic region.
- Tail dependence for the TP has increased sharply -> Sudden jumps in TP have become more synchronized across countries.

Increased correlation and synchronized jumps in yields -> diversification has become increasingly difficult.
The component of yields relevant for optimization is expected excess returns, which is related to TP (unobserved).
HF cycle share of TP variation declined recently while share of LF cycle increased -> tighter covariance among expected returns may reflect LF phenomena (LSAP, MP goals: IT), which has implications for investors w/ longer investment horizons (may not be well-diversified).

Propagation of MP shocks
Since comovement, spillover, tail dependence, MP international spillovers?
Response of global yields to US target shocks is strongly increasing w/ horizon. Delayed and protracted response of global yields to US target shocks (more than 1-to-1 w/ a substantial delay). 
Due to expectations of further Fed tightening or repricing of interest rate risk (higher TP)? 
ST increase strongly and persistently in response to a Fed tightening -> Investors expect global CB to follow Fed's decision (not due to FX movements).
Sharp reaction of global yields to US target shocks is driven by a reassessment of global policy rate expectations (ST) not by a repricing of global interest rate risk (TP).
US path and TP shocks lead to substantial adjustments of global yields through TP components.

Alternative lens for the propagation: Betas estimated in the frequency domain help determine extent to which lead-lag relationships persist. 
If shocks are sluggish and gather momentum, LF should have meaningful power to explain covariance b/w front and back end of YC.
- Betas of ST increase from HF to LF consistent w/ shocks being persistent (ST effect increasing w/ horizon) but modest contribution of LF to total variation (considerable HF noise in transmission toward the back of the YC).
- Betas of TP do not increase with frequency consistent w/ transmission from the US policy rate path to revisions in foreign anticipated policy paths (not revisions in TP).
- These mechanisms might not be very stable.


Dahlquist \& Hasseltoft (2016)
Excess return predictability and time-varying risk premia are two sides of the same coin.
Rational variation in risk premia arise from either: time-varyings economic risks or time-varying risk aversion among investors.
Future excess returns can be predictable even if markets are efficient: investors require higher risk premia during economic recessions.
The expectations hypothesis (zero or constant risk premia) is rejected in international data.
- The is significant time variation in international bond risk premia.
CP regressions predict excess returns across countries. 
- Predictive performance deteriorated in GFC. Reasons: UMP, fiscal policy uncertainty, time variation in funding and liquidity conditions, credit risk.
A global forecasting factor predicts local bond excess returns significantly -> commonality in risk premia across international bond markets.
- Global and systematic factors drive changes in the riskiness of nominal bonds.
Global factor is closely related to US bond risk premia -> US risk premia matter for international bond returns.
The dynamics of the global factor are countercyclical and predict future economic growth.
- A rise in global bond risk premia is contemporaneously associated with bad economic times but signals improved times ahead.
Bond risk premia turn negative during some periods.
Negative bond risk premia: nominal bonds act as hedging assets rather than risky assets.
No evidence of state dependency in bond markets: global factor predicts returns across business cycles (not only in recessions).
Nominal bonds are risky assets. Theoretical explanations of time variation in bond risk premia (predictable excess returns): habit formation generate time-varying investors' risk aversion (Buraschi and Jiltsov 2007), time-varying long-run economic risk model (Bansal and Shaliastovich 2013), model in which the severity of disaster risk varies over time (Gabaix 2012). (Piazzesi and Schneider 2009).
Portfolio rebalancing theory: changes in the supply of government debt affect the level and slope of the yield curve (Krishnamurty and Vissing-Jorgensen 2011, 2012). Preferred-habitat theory: increase supply of LT government debt steepens the yield curve and predicts excess returns on LT bonds (Greenwood and Vayanos 2014).


Jotikasthira, Le and Lundblad (2015)
Yield curve fluctuations across different currencies are highly correlated.
Channels through which macroeconomic shocks are transmitted across borders.
A world inflation factor drives risk compensation for long-term bonds.


Driessen, Melenberg and Nijman (2003)
A world level factor explains approximately 50\% oof the variation in international bond returns.


Meldrum, Raczko and Spencer (2018)
Joint modelling does not bring any material benefits in capturing the dynamics of bond yields.
Joint models of US and German nominal yields do not offer economically significant advantages in fitting the cross-section of yields or predicting future yields. 
Joint models of US nominal and real yields do not offer economically significant advantages in fitting the cross-section of yields or predicting future yields.





