\begin{abstract}
	This paper documents the channels through which U.S. monetary policy impacts the sovereign bond yields of emerging markets.
	Traditional decompositions of sovereign yields are not suitable for emerging markets because they rely on a default-free assumption. 
	Instead, I decompose the yields of 15 emerging markets into average expected future short-term interest rates, a term premium and compensation for credit risk.
	I use this decomposition to analyze the transmission channels of U.S. monetary policy surprises identified with intraday data.
	I find that the response of emerging market yields to target, forward guidance and asset purchase surprises is economically significant, yet delayed over days. 
	In addition, unanticipated U.S. monetary policy decisions lead to a reassessment of policy rate expectations and a repricing of interest and credit risks in emerging markets. 
	Finally, U.S. unconventional monetary policies limit the monetary autonomy of emerging markets along the yield curve.
	
	% Find the word count in the terminal: pbpaste | wc -w
	\vspace{0.5cm}
	\noindent \textit{Keywords}: Credit risk, term premium, synthetic yields, emerging markets, affine term structure models, monetary policy spillovers.
	
	\vspace{0.2cm}
	\noindent \textit{JEL Classification}: E43, F34, G12, G15, H63.
	
	\vfill
	
	\pagebreak
\end{abstract}

%Second, the surprises spill over to all yield components; unanticipated U.S. monetary policy decisions therefore lead to a reassessment of policy rate expectations and a repricing of interest and credit risks in emerging markets.

%This paper documents the channels through which U.S. monetary policy affects the bond yields of emerging markets.
%I show that traditional decompositions of sovereign yields are not suitable for emerging markets, they rely on a default-free assumption. 
%Instead, I decompose the yields of 15 emerging markets into an expected future short-term interest rate, a term premium and a compensation for credit risk---the spread between the nominal and synthetic local currency yields.
%I use this decomposition to analyze the transmission channels of U.S. monetary policy surprises identified with intraday data.
%The response of yields at different maturities is strong, yet delayed, and depends on the type of news. 
%Easing surprises in the policy rate are primarily transmitted through lower expected short rates, whereas the response to forward guidance and asset purchase easing surprises is mainly driven by a lower term premium; the effect on the credit risk compensation is usually temporary.
%U.S. monetary policy therefore leads to a reassessment of policy rate expectations and a repricing of (interest and credit) risks in emerging markets.

%The effects on / reaction of the credit risk compensation is temporary, except for news on asset purchases.
%the credit risk compensation also declines, except for asset purchases.

%I document strong, yet delayed, responses of the yields to U.S. monetary policy surprises identified with intraday data.
%Further, these responses, however, depend on the type of news. 

%using a new dataset of nominal and synthetic local currency yields, survey forecasts and U.S. monetary policy surprises identified with intraday data.

%This paper studies the dynamics of the sovereign yields of emerging markets and how they respond to U.S. monetary policy.
%Decomposing sovereign yields provides valuable information but relies on a default-free assumption. 
%Traditional decompositions are thus not suitable for emerging markets.
%I show that their yields can be decomposed into an expected future short-term interest rate, a term premium and a compensation for credit risk using a new dataset of nominal and synthetic local currency yields along with survey forecasts for 15 emerging markets from 2000 to 2019.
%I document a strong, yet delayed, response of the yields to U.S. monetary policy changes. 
%The decomposition reveals that those changes lead to a reassessment of policy rate expectations and a repricing of interest and credit risks in emerging markets.

%Sovereign yield curve decompositions provide valuable information.
%Yet they rely on a risk-free assumption.
%For emerging markets, traditional decompositions are biased because credit risk is not zero.
%I show that their yields can be decomposed into an expected future short-term interest rate, a term premium and a credit risk premium using a new dataset of nominal and synthetic local currency yields along with survey forecasts for 15 emerging markets from 2000 to 2019.
%Even though the components are weakly connected, they react strongly to monetary policy changes in the U.S.
%The decomposition thus reveals that U.S. monetary policy shocks lead to a reassessment of policy rate expectations and a repricing of interest and credit risks in emerging markets.

%They rely on a risk-free assumption, yet not all sovereign bonds are risk free.
%Credit risk in emerging market yields is not zero, so traditional decompositions for them are biased.
%All of the components are weakly connected but react strongly to monetary policy changes in the U.S.

%	This paper documents a strong and persistent response of the sovereign bond yields of emerging markets to U.S. monetary policy, despite a moderate initial reaction.
%	I characterize the response based on a novel decomposition of the yields into an expected future short-term interest rate, a term premium and a credit risk premium using a new dataset of nominal and synthetic local currency yields along with survey forecasts for 15 emerging markets from 2000 to 2019.
%	I find that monetary policy changes in the U.S. influence all three yield components.
%	Specifically, investors expect central banks in emerging markets to follow the monetary stance in the U.S.
%	Moreover, depending on the type of news, investors adjust not only the compensation they demand to hold long-term bonds but also the compensation against default.
%	U.S. monetary policy has therefore monetary as well as fiscal implications for emerging markets.

%but they rely on a risk-free assumption. Yet, not all sovereign bonds are risk free.
%I propose a novel decomposition of their yields into 

%	Furthermore even though the yields of emerging markets are less globally connected than those of advanced economies their response to U.S. monetary policy lasts longer.
% and thus gives rise to a reassessment of policy rate expectations and a repricing of interest and credit risks in those countries, 
% The delayed response is consistent with a portfolio rebalancing channel.

%This paper studies the response of the sovereign bond yields of emerging markets to U.S. monetary policy.
%To better characterize those responses, I propose a 
%This decomposition actually provides several new insights about the dynamics of bond yields in emerging markets.

%The analysis further reveals a trade-off between explicit and implicit default.
%that is region-specific.
%	I show that U.S. monetary policy not only influences each of the components of emerging market yields but the effect lasts longer than for advanced economies.
%	In addition to helping understand the yield responses
%	I first construct a new dataset of nominal and synthetic local currency yields along with survey forecasts for 15 emerging markets, and then estimate their yield curves using affine term structure models.
	
%	Bond risk premia in advanced countries is usually associated with a term premium, a compensation demanded by investors for bearing interest rate risk.
%	For emerging markets, however, the two concepts are different.
%	To show this, I estimate affine term structure models using a new dataset of synthetic local currency yields and survey forecasts.
%	This allows me to decompose the sovereign bond yields of 15 emerging markets into an expected future short-term interest rate and a bond risk premium, which consists of a credit risk premium and a pure term premium.
%	I then discuss several new insights about the dynamics of bond yields in emerging markets, including a trade-off between explicit and implicit defaults, the levels of their long-term real interest rates, the response of their term premia to U.S. quantitative easing announcements and the asymmetric effects of global factors on their yield curves.

%	addressing (which boil down to) the drivers of the BRP and the international spillovers  of the U.S. monetary policy. First, BRP is driven by global and domestic factors, the role of... being relatively larger. Second, the yield components of EMs react to US MPS asymmetrically throughout the yield curve.
% After presenting some stylized facts about the components of their yields, I show that... Finally, I document that...

%a bond risk premium, which is further disaggregated into a credit risk premium and a pure term premium.
%	the negative relationship between the two components of their bond risk premia, 
%	This decomposition leverages on synthetic local currency yields and survey data.
%	The main component of long-term yields is the expected short rate, followed by the term premium, whose size more than doubles the credit risk premium. 
%	This decomposition provides new insights into the dynamics of bond yields in emerging markets.
%First, their term premia declined in response to U.S. quantitative easing announcements.
%Second, the levels of their long-term expected real interest rates are similar to those in advanced countries.
%Finally, there is less comovement in the components of the yields of emerging markets relative to advanced countries.
%	I overcome two common concerns when analyzing these yields by using synthetic local currency yield curves to account for credit risk and survey forecasts to address the small sample problem.
%	 and their risk premia is time-varying.
%	The comovement is mainly driven by the 
%	In fact, the term premia is more globally connected than the credit risk premia and the expected short rates.
%	Further, the global component of term premia is highly linked to the U.S. term premia, whereas its idiosyncratic component is countercyclical. 
%	Second, [a recent reduction in term premia owes in part to declining inflation uncertainty among emerging markets.]
%	Finally, U.S. monetary policy shocks mainly affect component(s): [1, 2, 3].