\begin{abstract}
	The sovereign debt of advanced economies is generally considered free of default risk; this is key in the estimation of their term premia. The risk-free assumption, however, is not appropriate for the sovereign debt of emerging market economies. I use yield curves adjusted for credit risk in order to adequately estimate the term premia of 15 emerging markets. 
	I find that, on average, a quarter of the 10-year yield of emerging economies is due to the term premium, more than double the size of the default premium previously reported in the literature.
	Further, the evidence shows that both global and domestic factors are important drivers of the term premia in emerging markets.
	%	I find that the main component for the 10-year yield of emerging markets is the expected future path of the short-term interest rate, while for advanced economies the main component is the term premium. 
	
	\noindent
	\textit{Keywords}: Synthetic yield curves, term premium, affine term structure models.
	
	\noindent
	\textit{JEL Classification}: E43, F34, G12, G15, H63.
	
	%\pagebreak
\end{abstract}