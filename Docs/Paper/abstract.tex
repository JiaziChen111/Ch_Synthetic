\begin{abstract}
	This paper studies whether and to what extent the sovereign yields of emerging markets are interconnected.
	I use synthetic local currency yield curves to account for credit risk in order to estimate the term premia and the expected short rates of 15 emerging markets,
	and show that a quarter of the 10-year yield of emerging economies is due to the term premium, more than double the size of the credit risk premium.
	I document that the bond yields of emerging markets [comove mainly driven by the term premia, which is more globally connected than the credit risk premia and the expected short rates.]
	[Further, the global component of term premia is highly linked to the U.S. term premia, whereas its idiosyncratic component is countercyclical.] 
	[Finally, a recent reduction in term premia owes in part to declining inflation uncertainty among emerging markets.]
	
%	Local currency bond yields are decomposed into three components.
%	The sovereign yields of emerging markets include a credit risk premium, even for bonds denominated in local currency.
%	Yield curve decompositions applied to advanced economies rely on the no-credit-risk assumption and are thus not well suited for emerging markets.
% The sovereign debt of advanced economies is generally considered free of default risk; this is key in the estimation of their term premia. The risk-free assumption, however, is not appropriate for the sovereign debt of emerging market economies.
% I find that the main component for the 10-year yield of emerging markets is the expected future path of the short-term interest rate, while for advanced economies the main component is the term premium. 
% Further, the evidence shows that both global and domestic factors are important drivers of the term premia in emerging markets.
	
	% Find the word count in the terminal: pbpaste | wc -w
	\vspace{0.5cm}
	\noindent \textit{Keywords}: Synthetic yield curves, term premium, credit risk, affine term structure models, international spillovers.
	
	\vspace{0.2cm}
	\noindent \textit{JEL Classification}: E43, F34, G12, G15, H63.
	
	\vfill
	
	\pagebreak
\end{abstract}