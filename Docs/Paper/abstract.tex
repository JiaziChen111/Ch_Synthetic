\begin{abstract}
	This paper decomposes the sovereign bond yields of 15 emerging markets.
	Two common concerns in analyzing these yields are credit risk and small sample sizes.
	I overcome these concerns by using synthetic local currency yield curves to account for credit risk and survey forecasts to address the small sample problem.
	This allows me to decompose the bond yields into an expected future short-term interest rate, a term premium and a credit risk premium.
	I exploit this decomposition to analyze the dynamics of bond yields in emerging markets.
	The dynamics of the risk premium are driven by both the term premium and the credit risk premium, but the size of the former more than doubles the latter.
	There is a declining trend in term premia that accelerated after the large-scale asset purchases announced in the U.S.
	There is also a downward trend in the long-term expected real rates of some emerging markets.
	Finally, bond yields in these countries comove but to a lesser degree than the bond yields of advanced countries.
%	 and their risk premia is time-varying.
%	The comovement is mainly driven by the 
%	In fact, the term premia is more globally connected than the credit risk premia and the expected short rates.
%	Further, the global component of term premia is highly linked to the U.S. term premia, whereas its idiosyncratic component is countercyclical. 
%	Second, [a recent reduction in term premia owes in part to declining inflation uncertainty among emerging markets.]
%	Finally, U.S. monetary policy shocks mainly affect component(s): [1, 2, 3].
	
	% Find the word count in the terminal: pbpaste | wc -w
	\vspace{0.5cm}
	\noindent \textit{Keywords}: Synthetic yield curves, term premium, credit risk, emerging markets, affine term structure models, international spillovers.
	
	\vspace{0.2cm}
	\noindent \textit{JEL Classification}: E43, F34, G12, G15, H63.
	
	\vfill
	
	\pagebreak
\end{abstract}