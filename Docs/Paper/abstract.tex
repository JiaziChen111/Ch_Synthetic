\begin{abstract}
	Bond risk premia in advanced countries is usually associated with a term premium, a compensation demanded by investors for bearing interest rate risk.
	For emerging markets, however, the two concepts are different.
	To show this, I estimate affine term structure models using a new dataset of synthetic local currency yields and survey forecasts.
	This allows me to decompose the sovereign bond yields of 15 emerging markets into an expected future short-term interest rate and a bond risk premium, which consists of a credit risk premium and a pure term premium.
	I then discuss several new insights about the dynamics of bond yields in emerging markets, including a trade-off between explicit and implicit defaults, the levels of their long-term real interest rates, the response of their term premia to U.S. quantitative easing announcements and the asymmetric effects of global factors on their yield curves.
%	addressing (which boil down to) the drivers of the BRP and the international spillovers  of the U.S. monetary policy. First, BRP is driven by global and domestic factors, the role of... being relatively larger. Second, the yield components of EMs react to US MPS asymmetrically throughout the yield curve.
% After presenting some stylized facts about the components of their yields, I show that... Finally, I document that...

%a bond risk premium, which is further disaggregated into a credit risk premium and a pure term premium.
%	the negative relationship between the two components of their bond risk premia, 
%	This decomposition leverages on synthetic local currency yields and survey data.
%	The main component of long-term yields is the expected short rate, followed by the term premium, whose size more than doubles the credit risk premium. 
%	This decomposition provides new insights into the dynamics of bond yields in emerging markets.
%First, their term premia declined in response to U.S. quantitative easing announcements.
%Second, the levels of their long-term expected real interest rates are similar to those in advanced countries.
%Finally, there is less comovement in the components of the yields of emerging markets relative to advanced countries.
%	I overcome two common concerns when analyzing these yields by using synthetic local currency yield curves to account for credit risk and survey forecasts to address the small sample problem.
%	 and their risk premia is time-varying.
%	The comovement is mainly driven by the 
%	In fact, the term premia is more globally connected than the credit risk premia and the expected short rates.
%	Further, the global component of term premia is highly linked to the U.S. term premia, whereas its idiosyncratic component is countercyclical. 
%	Second, [a recent reduction in term premia owes in part to declining inflation uncertainty among emerging markets.]
%	Finally, U.S. monetary policy shocks mainly affect component(s): [1, 2, 3].
	
	% Find the word count in the terminal: pbpaste | wc -w
	\vspace{0.5cm}
	\noindent \textit{Keywords}: Synthetic yields, term premium, credit risk, emerging markets, affine term structure models, international spillovers.
	
	\vspace{0.2cm}
	\noindent \textit{JEL Classification}: E43, F34, G12, G15, H63.
	
	\vfill
	
	\pagebreak
\end{abstract}